\documentclass[journal,12pt,twocolumn]{IEEEtran}

\usepackage{setspace}
\usepackage{gensymb}

\singlespacing


\usepackage[cmex10]{amsmath}

\usepackage{amsthm}

\usepackage{mathrsfs}
\usepackage{txfonts}
\usepackage{stfloats}
\usepackage{bm}
\usepackage{cite}
\usepackage{cases}
\usepackage{subfig}

\usepackage{longtable}
\usepackage{multirow}

\usepackage{enumitem}
\usepackage{mathtools}
\usepackage{steinmetz}
\usepackage{tikz}
\usepackage{circuitikz}
\usepackage{verbatim}
\usepackage{tfrupee}
\usepackage[breaklinks=true]{hyperref}

\usepackage{tkz-euclide}

\usetikzlibrary{calc,math}
\usepackage{listings}
    \usepackage{color}                                            %%
    \usepackage{array}                                            %%
    \usepackage{longtable}                                        %%
    \usepackage{calc}                                             %%
    \usepackage{multirow}                                         %%
    \usepackage{hhline}                                           %%
    \usepackage{ifthen}                                           %%
    \usepackage{lscape}     
\usepackage{multicol}
\usepackage{chngcntr}

\DeclareMathOperator*{\Res}{Res}

\renewcommand\thesection{\arabic{section}}
\renewcommand\thesubsection{\thesection.\arabic{subsection}}
\renewcommand\thesubsubsection{\thesubsection.\arabic{subsubsection}}

\renewcommand\thesectiondis{\arabic{section}}
\renewcommand\thesubsectiondis{\thesectiondis.\arabic{subsection}}
\renewcommand\thesubsubsectiondis{\thesubsectiondis.\arabic{subsubsection}}


\hyphenation{op-tical net-works semi-conduc-tor}
\def\inputGnumericTable{}                                 %%

\lstset{
%language=C,
frame=single, 
breaklines=true,
columns=fullflexible
}
\begin{document}


\newtheorem{theorem}{Theorem}[section]
\newtheorem{problem}{Problem}
\newtheorem{proposition}{Proposition}[section]
\newtheorem{lemma}{Lemma}[section]
\newtheorem{corollary}[theorem]{Corollary}
\newtheorem{example}{Example}[section]
\newtheorem{definition}[problem]{Definition}

\newcommand{\BEQA}{\begin{eqnarray}}
\newcommand{\EEQA}{\end{eqnarray}}
\newcommand{\define}{\stackrel{\triangle}{=}}
\bibliographystyle{IEEEtran}
\providecommand{\mbf}{\mathbf}
\providecommand{\pr}[1]{\ensuremath{\Pr\left(#1\right)}}
\providecommand{\qfunc}[1]{\ensuremath{Q\left(#1\right)}}
\providecommand{\sbrak}[1]{\ensuremath{{}\left[#1\right]}}
\providecommand{\lsbrak}[1]{\ensuremath{{}\left[#1\right.}}
\providecommand{\rsbrak}[1]{\ensuremath{{}\left.#1\right]}}
\providecommand{\brak}[1]{\ensuremath{\left(#1\right)}}
\providecommand{\lbrak}[1]{\ensuremath{\left(#1\right.}}
\providecommand{\rbrak}[1]{\ensuremath{\left.#1\right)}}
\providecommand{\cbrak}[1]{\ensuremath{\left\{#1\right\}}}
\providecommand{\lcbrak}[1]{\ensuremath{\left\{#1\right.}}
\providecommand{\rcbrak}[1]{\ensuremath{\left.#1\right\}}}
\theoremstyle{remark}
\newtheorem{rem}{Remark}
\newcommand{\sgn}{\mathop{\mathrm{sgn}}}
\providecommand{\abs}[1]{\left\vert#1\right\vert}
\providecommand{\res}[1]{\Res\displaylimits_{#1}} 
\providecommand{\norm}[1]{\left\lVert#1\right\rVert}
%\providecommand{\norm}[1]{\lVert#1\rVert}
\providecommand{\mtx}[1]{\mathbf{#1}}
\providecommand{\mean}[1]{E\left[ #1 \right]}
\providecommand{\fourier}{\overset{\mathcal{F}}{ \rightleftharpoons}}
%\providecommand{\hilbert}{\overset{\mathcal{H}}{ \rightleftharpoons}}
\providecommand{\system}{\overset{\mathcal{H}}{ \longleftrightarrow}}
	%\newcommand{\solution}[2]{\textbf{Solution:}{#1}}
\newcommand{\solution}{\noindent \textbf{Solution: }}
\newcommand{\cosec}{\,\text{cosec}\,}
\providecommand{\dec}[2]{\ensuremath{\overset{#1}{\underset{#2}{\gtrless}}}}
\newcommand{\myvec}[1]{\ensuremath{\begin{pmatrix}#1\end{pmatrix}}}
\newcommand{\mydet}[1]{\ensuremath{\begin{vmatrix}#1\end{vmatrix}}}
\numberwithin{equation}{subsection}
\makeatletter
\@addtoreset{figure}{problem}
\makeatother
\let\StandardTheFigure\thefigure
\let\vec\mathbf
\renewcommand{\thefigure}{\theproblem}
\def\putbox#1#2#3{\makebox[0in][l]{\makebox[#1][l]{}\raisebox{\baselineskip}[0in][0in]{\raisebox{#2}[0in][0in]{#3}}}}
     \def\rightbox#1{\makebox[0in][r]{#1}}
     \def\centbox#1{\makebox[0in]{#1}}
     \def\topbox#1{\raisebox{-\baselineskip}[0in][0in]{#1}}
     \def\midbox#1{\raisebox{-0.5\baselineskip}[0in][0in]{#1}}
\vspace{3cm}
\title{Assignment 7}
\author{Yenigalla Samyuktha}
\maketitle
\newpage
\bigskip
\renewcommand{\thefigure}{\theenumi}
\renewcommand{\thetable}{\theenumi}
\begin{abstract}
This document explains the process of finding the distance between a given point and a plane using Singular Value Decomposition.
\end{abstract}
Download all latex-tikz codes from 
%
\begin{lstlisting}
https://github.com/EE20MTECH14019/EE5609/tree/master/Assignment_7
\end{lstlisting}
%
\section{Problem}
Let $\mathbb{F}$  be the field of complex numbers. Are the following two systems of linear equations equivalent? If so, express each equation in each system as a linear combination of equations in other system. First system of equations:
\begin{align}
2x_1+(-1+i)x_2+x_4=0\\
3x_2-2ix_3+5x_4=0
\end{align}
The second system of equations:
\begin{align}
(1+\frac{i}{2})x_1+8x_2-ix_3-x_4=0\\
\frac{2}{3}x_1-\frac{1}{2}x_2+x_3+7x_4=0
\end{align}
\section{Solution}
Let $\vec{R_1}$ and  $\vec{R_2}$ be the reduced row echelon forms of the  augumented matrices of the following systems of homogeneous equations respectively.
\begin{align}
\vec{A}\vec{X}=\vec{0}\label{f1}\\
\vec{B}\vec{X}=\vec{0}\label{f2}
\end{align}
Where $\vec{A}$ and $\vec{B}$ as follows
\begin{align}
\vec{A}=\myvec{2&-1+i&0&1\\0&3&-2i&5}\label{sys1}\\
\vec{B}=\myvec{1+\frac{i}{2}&8&-i&-1\\\frac{2}{3}&\frac{-1}{2}&1&7}\label{sys2}
\end{align}
On performing elementary row operations on $\eqref{sys1}$,
\begin{align}
\vec{R_1}=\vec{C}\vec{A}
\end{align}
where $\vec{C}$ is the product of all elementary matrices. Reducing the first system of linear equations, we get,
\begin{align}
\vec{C}=\myvec{1&\frac{1-i}{2}\\0&1}\myvec{1&0\\0&\frac{1}{3}}\myvec{\frac{1}{2}&0\\0&1}\\
\vec{R_1}=\myvec{1&0&\frac{-1-i}{3}&\frac{4}{3}-\frac{5i}{6}\\0&1&\frac{-2i}{3}&\frac{5}{3}}\label{R1}
\end{align}
On performing elementary row operations on $\eqref{sys2}$,
\begin{align}
\vec{R_2}=\vec{D}\vec{A}
\end{align}
where $\vec{D}$ is the product of all elementary matrices. Reducing the second system of linear equations, we get,
\begin{align}
\vec{D}=\myvec{\frac{4}{5}(1-\frac{i}{2})&0\\0&1}\myvec{1&0\\\frac{-2}{3}&1}\myvec{1&0\\0&\frac{-6(143+43i)}{4909}}\myvec{1&\frac{16(-2+i)}{5}\\0&1}
\end{align}
\begin{align}
\vec{R_2}=\myvec{1&0&\frac{6702}{4909}-\frac{708i}{4909}&\frac{46620}{4909}-\frac{1998i}{4909}\\\\0&1&\frac{-2(441+472i)}{4909}&\frac{-2(3283+1332i)}{4909}}\label{R2}
\end{align}
From the equations \eqref{R1} and \eqref{R2}, we can say that 
\begin{align}
\vec{R_1}\neq\vec{R_2}
\end{align}
Hence the given systems of linear equations are not equivalent.
\end{document}