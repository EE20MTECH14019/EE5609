\documentclass[journal,12pt,twocolumn]{IEEEtran}

\usepackage{setspace}
\usepackage{gensymb}

\singlespacing
\usepackage[cmex10]{amsmath}

\usepackage{amsthm}

\usepackage{mathrsfs}
\usepackage{txfonts}
\usepackage{stfloats}
\usepackage{bm}
\usepackage{cite}
\usepackage{cases}
\usepackage{subfig}
\usepackage{float}
\usepackage{longtable}
\usepackage{multirow}
\usepackage{caption}
%\usepackage[font=bf,labelfont=bf]{caption}

\usepackage{enumitem}
\usepackage{mathtools}
\usepackage{steinmetz}
\usepackage{tikz}
\usepackage{circuitikz}
\usepackage{verbatim}
\usepackage{tfrupee}
\usepackage[breaklinks=true]{hyperref}

\usepackage{tkz-euclide}

\usetikzlibrary{calc,math}
\usepackage{listings}
    \usepackage{color}                                            %%
    \usepackage{array}                                            %%
    \usepackage{longtable}                                        %%
    \usepackage{calc}                                             %%
    \usepackage{multirow}                                         %%
    \usepackage{hhline}                                           %%
    \usepackage{ifthen}                                           %%
    \usepackage{lscape}     
\usepackage{multicol}
\usepackage{chngcntr}


\DeclareMathOperator*{\Res}{Res}

\renewcommand\thesection{\arabic{section}}
\renewcommand\thesubsection{\thesection.\arabic{subsection}}
\renewcommand\thesubsubsection{\thesubsection.\arabic{subsubsection}}

\renewcommand\thesectiondis{\arabic{section}}
\renewcommand\thesubsectiondis{\thesectiondis.\arabic{subsection}}
\renewcommand\thesubsubsectiondis{\thesubsectiondis.\arabic{subsubsection}}
\numberwithin{table}{section}

\hyphenation{op-tical net-works semi-conduc-tor}
\def\inputGnumericTable{}                                 %%

\lstset{
%language=C,
frame=single, 
breaklines=true,
columns=fullflexible
}
\makeatletter
\renewcommand*\env@matrix[1][*\c@MaxMatrixCols c]{%
  \hskip -\arraycolsep
  \let\@ifnextchar\new@ifnextchar
  \array{#1}}
\makeatother
\begin{document}


\newtheorem{theorem}{Theorem}[section]
\newtheorem{problem}{Problem}
\newtheorem{proposition}{Proposition}[section]
\newtheorem{lemma}{Lemma}[section]
\newtheorem{corollary}[theorem]{Corollary}
\newtheorem{example}{Example}[section]
\newtheorem{definition}[problem]{Definition}

\newcommand{\BEQA}{\begin{eqnarray}}
\newcommand{\EEQA}{\end{eqnarray}}
\newcommand{\define}{\stackrel{\triangle}{=}}
\bibliographystyle{IEEEtran}
\providecommand{\mbf}{\mathbf}
\providecommand{\pr}[1]{\ensuremath{\Pr\left(#1\right)}}
\providecommand{\qfunc}[1]{\ensuremath{Q\left(#1\right)}}
\providecommand{\sbrak}[1]{\ensuremath{{}\left[#1\right]}}
\providecommand{\lsbrak}[1]{\ensuremath{{}\left[#1\right.}}
\providecommand{\rsbrak}[1]{\ensuremath{{}\left.#1\right]}}
\providecommand{\brak}[1]{\ensuremath{\left(#1\right)}}
\providecommand{\lbrak}[1]{\ensuremath{\left(#1\right.}}
\providecommand{\rbrak}[1]{\ensuremath{\left.#1\right)}}
\providecommand{\cbrak}[1]{\ensuremath{\left\{#1\right\}}}
\providecommand{\lcbrak}[1]{\ensuremath{\left\{#1\right.}}
\providecommand{\rcbrak}[1]{\ensuremath{\left.#1\right\}}}
\theoremstyle{remark}
\newtheorem{rem}{Remark}
\newcommand{\sgn}{\mathop{\mathrm{sgn}}}
\providecommand{\abs}[1]{\left\vert#1\right\vert}
\providecommand{\res}[1]{\Res\displaylimits_{#1}} 
\providecommand{\norm}[1]{\left\lVert#1\right\rVert}
%\providecommand{\norm}[1]{\lVert#1\rVert}
\providecommand{\mtx}[1]{\mathbf{#1}}
\providecommand{\mean}[1]{E\left[ #1 \right]}
\providecommand{\fourier}{\overset{\mathcal{F}}{ \rightleftharpoons}}
%\providecommand{\hilbert}{\overset{\mathcal{H}}{ \rightleftharpoons}}
\providecommand{\system}{\overset{\mathcal{H}}{ \longleftrightarrow}}
	%\newcommand{\solution}[2]{\textbf{Solution:}{#1}}
\newcommand{\solution}{\noindent \textbf{Solution: }}
\newcommand{\cosec}{\,\text{cosec}\,}
\providecommand{\dec}[2]{\ensuremath{\overset{#1}{\underset{#2}{\gtrless}}}}
\newcommand{\myvec}[1]{\ensuremath{\begin{pmatrix}#1\end{pmatrix}}}
\newcommand{\mydet}[1]{\ensuremath{\begin{vmatrix}#1\end{vmatrix}}}
\numberwithin{equation}{subsection}
\makeatletter
\@addtoreset{figure}{problem}
\makeatother
\let\StandardTheFigure\thefigure
\let\vec\mathbf
\renewcommand{\thefigure}{\theproblem}
\def\putbox#1#2#3{\makebox[0in][l]{\makebox[#1][l]{}\raisebox{\baselineskip}[0in][0in]{\raisebox{#2}[0in][0in]{#3}}}}
     \def\rightbox#1{\makebox[0in][r]{#1}}
     \def\centbox#1{\makebox[0in]{#1}}
     \def\topbox#1{\raisebox{-\baselineskip}[0in][0in]{#1}}
     \def\midbox#1{\raisebox{-0.5\baselineskip}[0in][0in]{#1}}
\vspace{3cm}
\title{Assignment 11}
\author{Yenigalla Samyuktha\\EE20MTECH14019}
\maketitle
\newpage
\bigskip
\renewcommand{\thefigure}{\theenumi}
\renewcommand{\thetable}{\theenumi}
\begin{abstract}
This document explains a proof on linear transformations.
\end{abstract}
Download all latex-tikz codes from 
%
\begin{lstlisting}
https://github.com/EE20MTECH14019/EE5609/tree/master/Assignment_11
\end{lstlisting}
%
\section{Problem}
Let $\mathbb{W}$ be the set of all $2\times2$ complex Hermitian matrices, that is the sset of $2\times2$ complex matrices $\vec{A}$ ssuch that $\vec{A}_{ij}=\vec{\overline{A_{ji}}}$ (the bar denoting complex conjugation). $\mathbb{W}$ is a vector space over the field of real numbers, under the usual operations. Verify that
\begin{align}\label{iso}
\myvec{x\\y\\z\\t} \rightarrow \myvec{t+x&y+iz\\y-iz&t-x}
\end{align}
is an isomorphism of $\mathbb{R}^4$ onto $\mathbb{W}$.
\section{Solution}
\begin{enumerate}
\item \textbf{Check for linearity:}
The transformation T is given by
\begin{align}
T:\mathbb{R}^4\rightarrow\mathbb{W}\\
T\myvec{x\\y\\z\\t}=\myvec{t+x&y+iz\\y-iz&t-x}
\end{align}
Let vectors $\myvec{x_1\\y_1\\z_1\\t_1}$, $\myvec{x_2\\y_2\\z_2\\t_2} \in \mathbb{R}^4$ and scalar $\alpha\in\mathbb{R}$.
\begin{align}
T\brak{\myvec{x_1\\y_1\\z_1\\t_1}+\myvec{x_2\\y_2\\z_2\\t_2}}
=T\brak{\myvec{x_1+x_2\\y_1+y_2\\z_1+z_2\\t_1+t_2}}\\
=\myvec{(t_1+t_2)+(x_1+x_2)&(y_1+y_2)+i(z_1+z_2)\\(y_1+y_2)-i(z_1+z_2)&(t_1+t_2)-(x_1+x_2)}\\
=\myvec{t_1+x_1&y_1+iz_1\\y_1-iz_1&t_1-x_1}+\myvec{t_2+x_2&y_2+iz_2\\y_2-iz_2&t_2-x_2}\\
=T\brak{\myvec{x_1\\y_1\\z_1\\t_1}}+T\brak{\myvec{x_2\\y_2\\z_2\\t_2}} \label{vecadd}
\end{align}
\begin{align}
T\brak{\alpha\myvec{x_1\\y_1\\z_1\\t_1}}=T\brak{\myvec{\alpha x_1\\\alpha y_1\\\alpha z_1\\\alpha t_1}}\\
=\myvec{\alpha(t_1+x_1)&\alpha(y_1+iz_1)\\\alpha(y_1-iz_1)&\alpha(t_1-x_1)}\\
=\alpha\myvec{t_1+x_1&y_1+iz_1\\y_1-iz_1&t_1-x_1}\\
=\alpha T\myvec{x_1\\y_1\\z_1\\t_1} \label{scalar}
\end{align}
From the equations \eqref{vecadd} and \eqref{scalar}, we can say that transformation T is linear.
\item \textbf{Check for invertibility of T: }
Consider the following matrix representation of the transformation T,
\begin{align}
\myvec{1&0&0&1\\0&1&i&0\\0&1&-i&0\\-0&0&0&1}\myvec{x\\y\\z\\t}=\myvec{x+t\\y+iz\\y-iz\\t-x} \label{tx}
\end{align}
We now form a $2\times2$ complex Hermitian matrix from the transformed complex vector in \eqref{tx} as follows.
\begin{align}
\myvec{t+x&y+iz\\y-iz&t-x}
\end{align}
Hence the transformation matrix $\vec{A}$ is,
\begin{align}\label{txA}
\vec{A}=\myvec{1&0&0&1\\0&1&i&0\\0&1&-i&0\\-0&0&0&1}
\end{align}
Consider the agumented matrix,
\begin{equation}\label{AI}
\vec{\brak{A|I}}=
  \begin{bmatrix}[cccc|cccc]
   1&0&0&1&1&0&0&0\\
   0&1&i&0&0&1&0&0\\
   0&1&-i&0&0&0&1&0\\
  -0&0&0&1&0&0&0&1\\
\end{bmatrix}
\end{equation}
Row-reducing the equation \eqref{AI}, we get
\begin{equation}\label{Ainv}
[\vec{I}|\vec{A}^{-1}]=
 \begin{bmatrix}[cccc|cccc]
 1&0&0&0&\frac{1}{2}&0&0&\frac{-1}{2}\\
 0&1&0&0&0&\frac{1}{2}&\frac{1}{2}&0\\
 0&0&1&0&0&\frac{-i}{2}&\frac{i}{2}0\\
 0&0&0&1&\frac{1}{2}&0&0&\frac{1}{2}\\
\end{bmatrix} 
\end{equation}
From the equation \eqref{Ainv}, we can say that the transformation matrix $\vec{A}$ is invertible.
Hence we can write the inverse transformation from $\mathbb{W}$ onto $\mathbb{R}^4$ as follows,
\begin{align}
\myvec{t+x&y+iz\\y-iz&t-x} \rightarrow \myvec{x\\y\\z\\t}
\end{align}
Where the matrix representation is,
\begin{align}\label{Tinv}
\myvec{\frac{1}{2}&0&0&\frac{-1}{2}\\0&\frac{1}{2}&\frac{1}{2}&0\\0&\frac{-i}{2}&\frac{i}{2}&0\\\frac{1}{2}&0&0&\frac{1}{2}}
\myvec{x+t\\y+iz\\y-iz\\t-x}=\myvec{x\\y\\z\\t}
\end{align}
\end{enumerate}
We know $\mathbb{R}^4$ is isomorphic to $\mathbb{W}$ if there exists
a linear transformation T : $\mathbb{R}^4 \rightarrow \mathbb{W}$ that is invertible. Such a T is
an isomorphism from $\mathbb{R}^4$ onto $\mathbb{W}$.
Hence from equations \eqref{tx} and \eqref{Tinv}, we can say that \eqref{iso} is isomorphism  from $\mathbb{R}^4$ onto $\mathbb{W}$.
\end{document}