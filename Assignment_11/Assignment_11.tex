\documentclass[journal,12pt,twocolumn]{IEEEtran}

\usepackage{setspace}
\usepackage{gensymb}

\singlespacing
\usepackage[cmex10]{amsmath}

\usepackage{amsthm}

\usepackage{mathrsfs}
\usepackage{txfonts}
\usepackage{stfloats}
\usepackage{bm}
\usepackage{cite}
\usepackage{cases}
\usepackage{subfig}
\usepackage{float}
\usepackage{longtable}
\usepackage{multirow}
\usepackage{caption}
%\usepackage[font=bf,labelfont=bf]{caption}

\usepackage{enumitem}
\usepackage{mathtools}
\usepackage{steinmetz}
\usepackage{tikz}
\usepackage{circuitikz}
\usepackage{verbatim}
\usepackage{tfrupee}
\usepackage[breaklinks=true]{hyperref}

\usepackage{tkz-euclide}

\usetikzlibrary{calc,math}
\usepackage{listings}
    \usepackage{color}                                            %%
    \usepackage{array}                                            %%
    \usepackage{longtable}                                        %%
    \usepackage{calc}                                             %%
    \usepackage{multirow}                                         %%
    \usepackage{hhline}                                           %%
    \usepackage{ifthen}                                           %%
    \usepackage{lscape}     
\usepackage{multicol}
\usepackage{chngcntr}


\DeclareMathOperator*{\Res}{Res}

\renewcommand\thesection{\arabic{section}}
\renewcommand\thesubsection{\thesection.\arabic{subsection}}
\renewcommand\thesubsubsection{\thesubsection.\arabic{subsubsection}}

\renewcommand\thesectiondis{\arabic{section}}
\renewcommand\thesubsectiondis{\thesectiondis.\arabic{subsection}}
\renewcommand\thesubsubsectiondis{\thesubsectiondis.\arabic{subsubsection}}
\numberwithin{table}{section}

\hyphenation{op-tical net-works semi-conduc-tor}
\def\inputGnumericTable{}                                 %%

\lstset{
%language=C,
frame=single, 
breaklines=true,
columns=fullflexible
}
\makeatletter
\renewcommand*\env@matrix[1][*\c@MaxMatrixCols c]{%
  \hskip -\arraycolsep
  \let\@ifnextchar\new@ifnextchar
  \array{#1}}
\makeatother
\begin{document}


\newtheorem{theorem}{Theorem}[section]
\newtheorem{problem}{Problem}
\newtheorem{proposition}{Proposition}[section]
\newtheorem{lemma}{Lemma}[section]
\newtheorem{corollary}[theorem]{Corollary}
\newtheorem{example}{Example}[section]
\newtheorem{definition}[problem]{Definition}

\newcommand{\BEQA}{\begin{eqnarray}}
\newcommand{\EEQA}{\end{eqnarray}}
\newcommand{\define}{\stackrel{\triangle}{=}}
\bibliographystyle{IEEEtran}
\providecommand{\mbf}{\mathbf}
\providecommand{\pr}[1]{\ensuremath{\Pr\left(#1\right)}}
\providecommand{\qfunc}[1]{\ensuremath{Q\left(#1\right)}}
\providecommand{\sbrak}[1]{\ensuremath{{}\left[#1\right]}}
\providecommand{\lsbrak}[1]{\ensuremath{{}\left[#1\right.}}
\providecommand{\rsbrak}[1]{\ensuremath{{}\left.#1\right]}}
\providecommand{\brak}[1]{\ensuremath{\left(#1\right)}}
\providecommand{\lbrak}[1]{\ensuremath{\left(#1\right.}}
\providecommand{\rbrak}[1]{\ensuremath{\left.#1\right)}}
\providecommand{\cbrak}[1]{\ensuremath{\left\{#1\right\}}}
\providecommand{\lcbrak}[1]{\ensuremath{\left\{#1\right.}}
\providecommand{\rcbrak}[1]{\ensuremath{\left.#1\right\}}}
\theoremstyle{remark}
\newtheorem{rem}{Remark}
\newcommand{\sgn}{\mathop{\mathrm{sgn}}}
\providecommand{\abs}[1]{\left\vert#1\right\vert}
\providecommand{\res}[1]{\Res\displaylimits_{#1}} 
\providecommand{\norm}[1]{\left\lVert#1\right\rVert}
%\providecommand{\norm}[1]{\lVert#1\rVert}
\providecommand{\mtx}[1]{\mathbf{#1}}
\providecommand{\mean}[1]{E\left[ #1 \right]}
\providecommand{\fourier}{\overset{\mathcal{F}}{ \rightleftharpoons}}
%\providecommand{\hilbert}{\overset{\mathcal{H}}{ \rightleftharpoons}}
\providecommand{\system}{\overset{\mathcal{H}}{ \longleftrightarrow}}
	%\newcommand{\solution}[2]{\textbf{Solution:}{#1}}
\newcommand{\solution}{\noindent \textbf{Solution: }}
\newcommand{\cosec}{\,\text{cosec}\,}
\providecommand{\dec}[2]{\ensuremath{\overset{#1}{\underset{#2}{\gtrless}}}}
\newcommand{\myvec}[1]{\ensuremath{\begin{pmatrix}#1\end{pmatrix}}}
\newcommand{\mydet}[1]{\ensuremath{\begin{vmatrix}#1\end{vmatrix}}}
\numberwithin{equation}{subsection}
\makeatletter
\@addtoreset{figure}{problem}
\makeatother
\let\StandardTheFigure\thefigure
\let\vec\mathbf
\renewcommand{\thefigure}{\theproblem}
\def\putbox#1#2#3{\makebox[0in][l]{\makebox[#1][l]{}\raisebox{\baselineskip}[0in][0in]{\raisebox{#2}[0in][0in]{#3}}}}
     \def\rightbox#1{\makebox[0in][r]{#1}}
     \def\centbox#1{\makebox[0in]{#1}}
     \def\topbox#1{\raisebox{-\baselineskip}[0in][0in]{#1}}
     \def\midbox#1{\raisebox{-0.5\baselineskip}[0in][0in]{#1}}
\vspace{3cm}
\title{Assignment 11}
\author{Yenigalla Samyuktha\\EE20MTECH14019}
\maketitle
\newpage
\bigskip
\renewcommand{\thefigure}{\theenumi}
\renewcommand{\thetable}{\theenumi}
\begin{abstract}
This document explains a proof on linear transformations.
\end{abstract}
Download all latex-tikz codes from 
%
\begin{lstlisting}
https://github.com/EE20MTECH14019/EE5609/tree/master/Assignment_11
\end{lstlisting}
%
\section{Problem}
Let $\mathbb{W}$ be the set of all $2\times2$ complex Hermitian matrices, that is the sset of $2\times2$ complex matrices $\vec{A}$ ssuch that $\vec{A}_{ij}=\vec{\overline{A_{ji}}}$ (the bar denoting complex conjugation). $\mathbb{W}$ is a vector space over the field of real numbers, under the usual operations. Verify that
\begin{align}\label{iso}
\myvec{x\\y\\z\\t} \rightarrow \myvec{t+x&y+iz\\y-iz&t-x}
\end{align}
is an isomorphism of $\mathbb{R}^4$ onto $\mathbb{W}$.
\section{Solution}
\begin{enumerate}
\item \textbf{Check for linearity:}
The transformation T is given by
\begin{align}
T\colon\mathbb{R}^4\to\mathbb{W}\\
T\myvec{x\\y\\z\\t}=\myvec{t+x&y+iz\\y-iz&t-x} \label{Trans}
\end{align}
Let $\vec{x}=\myvec{x\\y\\t\\z}$. Expressing R.H.S of equation \eqref{Trans} using Kronecker Product, 
\begin{align}
T\brak{\vec{x}}=\myvec{\myvec{1&0&0&1}\vec{x}&\myvec{0&1&i&0}\vec{x}\\\myvec{0&1&-i&0}\vec{x}&\myvec{-1&0&0&1}\vec{x}}\\
=\myvec{\myvec{1&0&0&1\\0&1&-i&0}\vec{x}&\myvec{0&1&i&0\\-1&0&0&1}\vec{x}}\\
=\myvec{1&0&0&1&0&1&i&0\\0&1&-i&0&-1&0&0&1}\myvec{x&0\\y&0\\z&0\\t&0\\0&x\\0&y\\0&z\\0&t}
\end{align}
\begin{align}
\implies T\brak{\vec{x}}=\myvec{\vec{A}&\vec{B}}\myvec{\vec{x}&\vec{0}_{4\times1}\\\vec{0}_{4\times1}&\vec{x}}\label{block}
\end{align}
Where $\vec{A}$ and $\vec{B}$ are block matrices.
\begin{align}
\vec{A}=\myvec{1&0&0&1\\0&1&-i&0}\\
\vec{B}=\myvec{0&1&i&0\\-1&0&0&1}
\end{align}
The Kronecker Product of $\vec{I}_2$ and $\vec{x}$ gives the block matrix in equation \eqref{block}.
\begin{align}
\vec{I}_{2\times2}\otimes\vec{x}_{4\times1}=\myvec{\vec{x}&\vec{0}\\\vec{0}&\vec{x}}_{8\times2}
\end{align}
Hence we can write equation \eqref{block} as,
\begin{align}
 T\brak{\vec{x}}=\myvec{\vec{A}&\vec{B}}\brak{\vec{I}\otimes\vec{x}} \label{final}
\end{align}
Let $\vec{x}_1,\vec{x}_2\in\mathbb{R}^4$ and $\alpha,\beta\in\mathbb{R}$.
\begin{align}
T\brak{\alpha\vec{x}_1+\beta\vec{x}_2}=\myvec{\vec{A}&\vec{B}}\brak{\vec{I}\otimes\brak{\alpha\vec{x}_1+\beta\vec{x}_2}}\\
=\alpha\myvec{\vec{A}&\vec{B}}\brak{\vec{I}\otimes\vec{x}_1}+\beta\myvec{\vec{A}&\vec{B}}\brak{\vec{I}\otimes\vec{x}_2}\\
=\alpha T\vec{x}_1+\beta T\vec{x}_2 \label{linear}
\end{align}
Therefore from equation \eqref{linear}, we can say T is linear transformation.
\item \textbf{Check for one-one property: }
For transformation T to be one-one, we can prove if $T\brak{\vec{x}}=\vec{0}$, that implies $\vec{x}=\vec{0}$. From the equation \eqref{final},
\begin{align}
T\brak{\vec{x}}=\vec{0}\label{1}\\
\myvec{\vec{A}&\vec{B}}\brak{\vec{I}\otimes\vec{x}}=\vec{0}
\end{align}
\begin{align}
\implies \myvec{1&0&0&1&0&1&i&0\\0&1&-i&0&-1&0&0&1}\myvec{x&0\\y&0\\z&0\\t&0\\0&x\\0&y\\0&z\\0&t}=\vec{0}_{2\times2}
\end{align}
From equation \eqref{Trans},
\begin{align}
\myvec{t+x&y+iz\\y-iz&t-x}=\vec{0}_{2\times2}\\
\implies x=0,y=0,z=0,t=0\\
\implies \vec{x}=\vec{0}\label{2}
\end{align}
Hence from \eqref{1} and \eqref{2}, T is one-one and that implies $T\colon\mathbb{R}^4\to\mathbb{W}$ is isomorphism.
\end{enumerate}
\end{document}