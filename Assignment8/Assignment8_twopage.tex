\documentclass[journal,12pt,twocolumn]{IEEEtran}

\usepackage{setspace}
\usepackage{gensymb}

\singlespacing


\usepackage[cmex10]{amsmath}

\usepackage{amsthm}

\usepackage{mathrsfs}
\usepackage{txfonts}
\usepackage{stfloats}
\usepackage{bm}
\usepackage{cite}
\usepackage{cases}
\usepackage{subfig}

\usepackage{longtable}
\usepackage{multirow}

\usepackage{enumitem}
\usepackage{mathtools}
\usepackage{steinmetz}
\usepackage{tikz}
\usepackage{circuitikz}
\usepackage{verbatim}
\usepackage{tfrupee}
\usepackage[breaklinks=true]{hyperref}

\usepackage{tkz-euclide}

\usetikzlibrary{calc,math}
\usepackage{listings}
    \usepackage{color}                                            %%
    \usepackage{array}                                            %%
    \usepackage{longtable}                                        %%
    \usepackage{calc}                                             %%
    \usepackage{multirow}                                         %%
    \usepackage{hhline}                                           %%
    \usepackage{ifthen}                                           %%
    \usepackage{lscape}     
\usepackage{multicol}
\usepackage{chngcntr}

\DeclareMathOperator*{\Res}{Res}

\renewcommand\thesection{\arabic{section}}
\renewcommand\thesubsection{\thesection.\arabic{subsection}}
\renewcommand\thesubsubsection{\thesubsection.\arabic{subsubsection}}

\renewcommand\thesectiondis{\arabic{section}}
\renewcommand\thesubsectiondis{\thesectiondis.\arabic{subsection}}
\renewcommand\thesubsubsectiondis{\thesubsectiondis.\arabic{subsubsection}}


\hyphenation{op-tical net-works semi-conduc-tor}
\def\inputGnumericTable{}                                 %%

\lstset{
%language=C,
frame=single, 
breaklines=true,
columns=fullflexible
}
\begin{document}


\newtheorem{theorem}{Theorem}[section]
\newtheorem{problem}{Problem}
\newtheorem{proposition}{Proposition}[section]
\newtheorem{lemma}{Lemma}[section]
\newtheorem{corollary}[theorem]{Corollary}
\newtheorem{example}{Example}[section]
\newtheorem{definition}[problem]{Definition}

\newcommand{\BEQA}{\begin{eqnarray}}
\newcommand{\EEQA}{\end{eqnarray}}
\newcommand{\define}{\stackrel{\triangle}{=}}
\bibliographystyle{IEEEtran}
\providecommand{\mbf}{\mathbf}
\providecommand{\pr}[1]{\ensuremath{\Pr\left(#1\right)}}
\providecommand{\qfunc}[1]{\ensuremath{Q\left(#1\right)}}
\providecommand{\sbrak}[1]{\ensuremath{{}\left[#1\right]}}
\providecommand{\lsbrak}[1]{\ensuremath{{}\left[#1\right.}}
\providecommand{\rsbrak}[1]{\ensuremath{{}\left.#1\right]}}
\providecommand{\brak}[1]{\ensuremath{\left(#1\right)}}
\providecommand{\lbrak}[1]{\ensuremath{\left(#1\right.}}
\providecommand{\rbrak}[1]{\ensuremath{\left.#1\right)}}
\providecommand{\cbrak}[1]{\ensuremath{\left\{#1\right\}}}
\providecommand{\lcbrak}[1]{\ensuremath{\left\{#1\right.}}
\providecommand{\rcbrak}[1]{\ensuremath{\left.#1\right\}}}
\theoremstyle{remark}
\newtheorem{rem}{Remark}
\newcommand{\sgn}{\mathop{\mathrm{sgn}}}
\providecommand{\abs}[1]{\left\vert#1\right\vert}
\providecommand{\res}[1]{\Res\displaylimits_{#1}} 
\providecommand{\norm}[1]{\left\lVert#1\right\rVert}
%\providecommand{\norm}[1]{\lVert#1\rVert}
\providecommand{\mtx}[1]{\mathbf{#1}}
\providecommand{\mean}[1]{E\left[ #1 \right]}
\providecommand{\fourier}{\overset{\mathcal{F}}{ \rightleftharpoons}}
%\providecommand{\hilbert}{\overset{\mathcal{H}}{ \rightleftharpoons}}
\providecommand{\system}{\overset{\mathcal{H}}{ \longleftrightarrow}}
	%\newcommand{\solution}[2]{\textbf{Solution:}{#1}}
\newcommand{\solution}{\noindent \textbf{Solution: }}
\newcommand{\cosec}{\,\text{cosec}\,}
\providecommand{\dec}[2]{\ensuremath{\overset{#1}{\underset{#2}{\gtrless}}}}
\newcommand{\myvec}[1]{\ensuremath{\begin{pmatrix}#1\end{pmatrix}}}
\newcommand{\mydet}[1]{\ensuremath{\begin{vmatrix}#1\end{vmatrix}}}
\numberwithin{equation}{subsection}
\makeatletter
\@addtoreset{figure}{problem}
\makeatother
\let\StandardTheFigure\thefigure
\let\vec\mathbf
\renewcommand{\thefigure}{\theproblem}
\def\putbox#1#2#3{\makebox[0in][l]{\makebox[#1][l]{}\raisebox{\baselineskip}[0in][0in]{\raisebox{#2}[0in][0in]{#3}}}}
     \def\rightbox#1{\makebox[0in][r]{#1}}
     \def\centbox#1{\makebox[0in]{#1}}
     \def\topbox#1{\raisebox{-\baselineskip}[0in][0in]{#1}}
     \def\midbox#1{\raisebox{-0.5\baselineskip}[0in][0in]{#1}}
\vspace{3cm}
\title{Assignment 8}
\author{Yenigalla Samyuktha\\EE20MTECH14019}
\maketitle
\newpage
\bigskip
\renewcommand{\thefigure}{\theenumi}
\renewcommand{\thetable}{\theenumi}
\begin{abstract}
This document lists out the axioms satisfied for a vector space.
\end{abstract}
Download all latex-tikz codes from 
%
\begin{lstlisting}
https://github.com/EE20MTECH14019/EE5609/tree/master/Assignment_8
\end{lstlisting}
%
\section{Problem}
On $\mathbb{R}^n$  define two operations
\begin{align}
\alpha\oplus\beta=\alpha-\beta\\
c\cdot\alpha=-c\alpha
\end{align}
The operations on the right are usual ones. Which of the axioms for a vector space are satisfied by $(\mathbb{R}^n,\oplus,\cdot)$?
\section{Solution}
Let $(\alpha,\beta,\gamma)\in \mathbb{R}^n$  and $c,c_1,c_2$ are scalars taken from the field $\mathbb{R}$ where the vector space is defined on. Table \ref{table:1} lists the axioms satisfied and not satisfied for $(\mathbb{R}^n,\oplus,\cdot)$.
\begin{table*}[h!]
\begin{center}
\begin{tabular}{|l|l|}
\hline
\textbf{UNSATISTIFD}&\textbf{SATISFIED}\\[0.5ex]\hline
\textbf{Associativity of addition}&\textbf{Additive identity}\\
$\alpha\oplus(\beta\oplus\gamma)=\alpha-\beta+\gamma$&
$\alpha\oplus\beta=\alpha-\beta=\alpha$\\
$(\alpha\oplus\beta)\oplus\gamma=\alpha-\beta-\gamma$&
\text{Additive identity is $\beta$}\\
$\alpha\oplus(\beta\oplus\gamma)\neq (\alpha\oplus\beta)\oplus\gamma$& \text{unique }$\beta=(0,0,....0)$\\ [0.5ex] \hline
\textbf{Commutativity of addition}& \textbf{Additive inverse}\\
$\alpha\oplus\beta=\alpha-\beta$ &  $\alpha\oplus\alpha=\alpha-\alpha=0$\\
$\beta\oplus\alpha=\beta-\alpha $&
\text{Additive inverse is $\alpha$}\\
$\alpha\oplus\beta\neq\beta\oplus\alpha$& \\[0.5ex]\hline
\textbf{Scalar multiplication with field multiplication}&\\
$(c_1c_2)\cdot\alpha=(-c_1c_2) \alpha$&\\
$c_1\cdot(c_2\cdot\alpha)=c_1c_2 \alpha$&\\
$(c_1c_2)\cdot\alpha\neq c_1\cdot(c_2\cdot\alpha)$&\\[0.5ex]\hline
\textbf{Identity element of scalar multiplication}&\\
$1\cdot\alpha=-\alpha=\alpha \text{ for } \alpha=(0,0,....,0)$&\\
$1\cdot\alpha=-\alpha\neq \alpha$ \text{ $\forall$ } $\alpha\neq(0,0,...,0) $
&\\[0.5ex] \hline
\textbf{Distributivity of scalar multiplication w.r.t vector addition}&\\
$c\cdot(\alpha\oplus\beta)=-c(\alpha-\beta)$&\\
$c\cdot\alpha\oplus c\cdot\beta=-c\alpha-(-c\beta)$&\\
$c\cdot(\alpha\oplus\beta)\neq c\cdot\alpha\oplus c\cdot\beta$&\\[0.5ex]\hline
\textbf{Distributivity of scalar multiplication w.r.t field addition}&\\
$(c_1+c_2)\cdot\alpha=-(c_1+c_2)\alpha$&\\
$c_1\cdot\alpha\oplus c_2\cdot\beta=-c_1\alpha-(-c_2\beta)$&\\
$(c_1+c_2)\cdot\alpha\neq c_1\cdot\alpha\oplus c_2\cdot\beta$&\\[0.5ex]\hline
\end{tabular}
\caption{Axioms of vector space $(\mathbb{R}^n,\oplus,\cdot)$ }
\label{table:1}
\end{center}
\vspace{-0.5cm}
\end{table*}
\end{document}