\documentclass[journal,12pt,twocolumn]{IEEEtran}

\usepackage{setspace}
\usepackage{gensymb}

\singlespacing
\usepackage[cmex10]{amsmath}

\usepackage{amsthm}
\usepackage{mathrsfs}
\usepackage{txfonts}
\usepackage{stfloats}
\usepackage{bm}
\usepackage{cite}
\usepackage{cases}
\usepackage{subfig}
\usepackage{float}
\usepackage{longtable}
\usepackage{multirow}
\usepackage{caption}
%\usepackage[font=bf,labelfont=bf]{caption}

\usepackage{enumitem}
\usepackage{mathtools}
\usepackage{steinmetz}
\usepackage{tikz}
\usepackage{circuitikz}
\usepackage{verbatim}
\usepackage{tfrupee}
\usepackage[breaklinks=true]{hyperref}

\usepackage{tkz-euclide}

\usetikzlibrary{calc,math}
\usepackage{listings}
    \usepackage{color}                                            %%
    \usepackage{array}                                            %%
    \usepackage{longtable}                                        %%
    \usepackage{calc}                                             %%
    \usepackage{multirow}                                         %%
    \usepackage{hhline}                                           %%
    \usepackage{ifthen}                                           %%
    \usepackage{lscape}     
\usepackage{multicol}
\usepackage{chngcntr}


\DeclareMathOperator*{\Res}{Res}

\renewcommand\thesection{\arabic{section}}
\renewcommand\thesubsection{\thesection.\arabic{subsection}}
\renewcommand\thesubsubsection{\thesubsection.\arabic{subsubsection}}

\renewcommand\thesectiondis{\arabic{section}}
\renewcommand\thesubsectiondis{\thesectiondis.\arabic{subsection}}
\renewcommand\thesubsubsectiondis{\thesubsectiondis.\arabic{subsubsection}}
\numberwithin{table}{section}

\hyphenation{op-tical net-works semi-conduc-tor}
\def\inputGnumericTable{}                                 %%

\lstset{
%language=C,
frame=single, 
breaklines=true,
columns=fullflexible
}
\makeatletter
\renewcommand*\env@matrix[1][*\c@MaxMatrixCols c]{%
  \hskip -\arraycolsep
  \let\@ifnextchar\new@ifnextchar
  \array{#1}}
\makeatother
\begin{document}


\newtheorem{theorem}{Theorem}[section]
\newtheorem{problem}{Problem}
\newtheorem{proposition}{Proposition}[section]
\newtheorem{lemma}{Lemma}[section]
\newtheorem{corollary}[theorem]{Corollary}
\newtheorem{example}{Example}[section]
\newtheorem{definition}[problem]{Definition}

\newcommand{\BEQA}{\begin{eqnarray}}
\newcommand{\EEQA}{\end{eqnarray}}
\newcommand{\define}{\stackrel{\triangle}{=}}
\bibliographystyle{IEEEtran}
\providecommand{\mbf}{\mathbf}
\providecommand{\pr}[1]{\ensuremath{\Pr\left(#1\right)}}
\providecommand{\qfunc}[1]{\ensuremath{Q\left(#1\right)}}
\providecommand{\sbrak}[1]{\ensuremath{{}\left[#1\right]}}
\providecommand{\lsbrak}[1]{\ensuremath{{}\left[#1\right.}}
\providecommand{\rsbrak}[1]{\ensuremath{{}\left.#1\right]}}
\providecommand{\brak}[1]{\ensuremath{\left(#1\right)}}
\providecommand{\lbrak}[1]{\ensuremath{\left(#1\right.}}
\providecommand{\rbrak}[1]{\ensuremath{\left.#1\right)}}
\providecommand{\cbrak}[1]{\ensuremath{\left\{#1\right\}}}
\providecommand{\lcbrak}[1]{\ensuremath{\left\{#1\right.}}
\providecommand{\rcbrak}[1]{\ensuremath{\left.#1\right\}}}
\theoremstyle{remark}
\newtheorem{rem}{Remark}
\newcommand{\sgn}{\mathop{\mathrm{sgn}}}
\providecommand{\abs}[1]{\left\vert#1\right\vert}
\providecommand{\res}[1]{\Res\displaylimits_{#1}} 
\providecommand{\norm}[1]{\left\lVert#1\right\rVert}
%\providecommand{\norm}[1]{\lVert#1\rVert}
\providecommand{\mtx}[1]{\mathbf{#1}}
\providecommand{\mean}[1]{E\left[ #1 \right]}
\providecommand{\fourier}{\overset{\mathcal{F}}{ \rightleftharpoons}}
%\providecommand{\hilbert}{\overset{\mathcal{H}}{ \rightleftharpoons}}
\providecommand{\system}{\overset{\mathcal{H}}{ \longleftrightarrow}}
	%\newcommand{\solution}[2]{\textbf{Solution:}{#1}}
\newcommand{\solution}{\noindent \textbf{Solution: }}
\newcommand{\cosec}{\,\text{cosec}\,}
\providecommand{\dec}[2]{\ensuremath{\overset{#1}{\underset{#2}{\gtrless}}}}
\newcommand{\myvec}[1]{\ensuremath{\begin{pmatrix}#1\end{pmatrix}}}
\newcommand{\mydet}[1]{\ensuremath{\begin{vmatrix}#1\end{vmatrix}}}
\numberwithin{equation}{subsection}
\makeatletter
\@addtoreset{figure}{problem}
\makeatother
\let\StandardTheFigure\thefigure
\let\vec\mathbf
\renewcommand{\thefigure}{\theproblem}
\def\putbox#1#2#3{\makebox[0in][l]{\makebox[#1][l]{}\raisebox{\baselineskip}[0in][0in]{\raisebox{#2}[0in][0in]{#3}}}}
     \def\rightbox#1{\makebox[0in][r]{#1}}
     \def\centbox#1{\makebox[0in]{#1}}
     \def\topbox#1{\raisebox{-\baselineskip}[0in][0in]{#1}}
     \def\midbox#1{\raisebox{-0.5\baselineskip}[0in][0in]{#1}}
\vspace{3cm}
\title{Assignment 14}
\author{Yenigalla Samyuktha\\EE20MTECH14019}
\maketitle
\newpage
\bigskip
\renewcommand{\thefigure}{\theenumi}
\renewcommand{\thetable}{1}
\setlength{\tabcolsep}{20pt}
\renewcommand{\arraystretch}{1.5}
\begin{abstract}
This document explains a linear transmormation on polynomials.
\end{abstract}
Download all latex-tikz codes from 
%
\begin{lstlisting}
https://github.com/EE20MTECH14019/EE5609/tree/master/Assignment_14
\end{lstlisting}
%
\section{Problem}
Let $\mathbb{F}$ be a subfield of complex numbers and let T be the transformation on $\mathbb{F}\brak{x}$ defined by
\begin{align}
T\brak{\sum \limits_{i=0}^n c_i x^i}=\sum \limits_{i=0}^n \frac{c_i}{i+1} x^{i+1}\label{q}
\end{align}
Show that T is a non-singular linear operator on $\mathbb{F}\sbrak{x}$. Also show that T is not invertible.
\section{Solution}
The transformation T does integral of a polynomial. Table \ref{table:2} provides proof that the transformation T is a linear operator and non-singular. Table \ref{table:3} provides proof that T is not invertible, however there exists a left inverse. The parameters used in the proof are listed in the table \ref{table:1}.
\renewcommand{\thetable}{1}
\begin{table*}[ht!]
\begin{center}
\begin{tabular}{|l|l|}
\hline
\textbf{PARAMETER} & \textbf{DESCRIPTION}\\[0.5ex]
\hline
$\mathbb{F}$ & Field of complex numbers\\[0.5ex]
\hline
$\mathbb{F}^{n+1}$ & Vector space defined on the field $\mathbb{F}$\\[0.5ex]
\hline
$\mathbb{F}\sbrak{x}$ & Subspce of $\mathbb{F}^{n+1}$ \\[0.5ex] 
\hline
$\cbrak{1,x,x^2,x^3,\hdots,x^{n+1}}$ & Basis for $\mathbb{F}\sbrak{x}$ \\[0.5ex] 
\hline
$T\colon \mathbb{F}\sbrak{x} \rightarrow \mathbb{F}\sbrak{x}$ & Transformation T\\[0.5ex] 
\hline
$\mathit{f}=\sum \limits_{i=0}^n c_i x^i$ & Polynomial $\mathit{f} \in \mathbb{F}\sbrak{x}$\\[0.5ex] 
\hline
$\mathit{f^\prime}=\sum \limits_{i=0}^n c^\prime_i x^i$ & Polynomial $\mathit{f^\prime} \in \mathbb{F}\sbrak{x}$\\[0.5ex] 
\hline
$c_i,c^\prime_i \; \forall i=0,2,\hdots n$ & Scalars in $\mathbb{F}$ and coefficients of polynomials $\mathit{f}$ and $\mathit{f^\prime}$\\[0.5ex] 
\hline
$\vec{M}_T$ & Transformation matrix for T\\[0.5ex] 
\hline
$N\brak{T}$ & Null Space of T\\[0.5ex] 
\hline
\end{tabular}
\caption{Parameters}
\label{table:1}
\end{center}
\vspace{-0.5cm}
\end{table*}

\renewcommand{\thetable}{2}
\begin{table*}[ht!]
\begin{center}
\begin{tabular}{|l|l|}
\hline
\textbf{Statement} & \textbf{Derivation} \\[0.5ex]
\hline
$\mathit{f}=\sum \limits_{i=0}^n c_i x^i$ & $\mathit{f}=\myvec{c_0&c_1&c_2&\cdots&c_n}^T_{\brak{n+1}\times1}$
\\ [0.5ex] 
\hline
$T\sbrak{\mathit{f}}=\vec{M}_T\mathit{f}$ & $T\sbrak{\mathit{f}}=
\myvec{0&0&0&\cdots&0\\1&0&0&\cdots&0\\0&\frac{1}{2}&0&\cdots&0\\0&0&\frac{1}{3}&\cdots&0\\\vdots&\vdots&\vdots&\cdots&\vdots\\0&0&0&\cdots&\frac{1}{n+1}}_{\brak{n+2}\times\brak{n+1}}
\myvec{c_0\\c_1\\c_2\\\vdots\\c_n}_{\brak{n+1}\times1}
=\myvec{0\\c_0\\\frac{c_1}{2}\\\frac{c_2}{3}\\\vdots\\\frac{c_n}{n+1}}$
\\ [0.5ex] 
\hline
T is a linear operator & 
\parbox{10cm}{\begin{align}
T\sbrak{\alpha\mathit{f}+\mathit{f^\prime}}\\
=\vec{M}_T\brak{\alpha\mathit{f}+\mathit{f^\prime}}\\
=\myvec{0&0&0&\cdots&0\\1&0&0&\cdots&0\\0&\frac{1}{2}&0&\cdots&0\\0&0&\frac{1}{3}&\cdots&0\\\vdots&\vdots&\vdots&\cdots&\vdots\\0&0&0&\cdots&\frac{1}{n+1}}
\myvec{\alpha c_0+c_0^\prime\\\alpha c_1+c_1^\prime\\\alpha c_2+c_2^\prime\\\vdots\\\alpha c_n+c_n^\prime}\\
=\myvec{0\\\alpha c_0+c_0^\prime\\\frac{\alpha c_1+c_1^\prime}{2}\\\frac{\alpha c_2+c_2^\prime}{3}\\\vdots\\\frac{\alpha c_n+c_n^\prime}{n+1}}
=\alpha\myvec{0\\c_0\\\frac{c_1}{2}\\\frac{c_2}{3}\\\vdots\\\frac{c_n}{n+1}}+\myvec{0\\c_0^\prime\\\frac{c_1^\prime}{2}\\\frac{c_2^\prime}{3}\\\vdots\\\frac{c_n^\prime}{n+1}}\\
=\alpha T\sbrak{\mathit{f}}+T\sbrak{\mathit{f^\prime}}\\
\therefore T\sbrak{\alpha\mathit{f}+\mathit{f^\prime}}=\alpha T\sbrak{\mathit{f}}+T\sbrak{\mathit{f^\prime}}
\end{align}}
\\ [0.5ex] 
\hline
T is non-singular & \parbox{10cm}{\begin{align}
T\sbrak{\mathit{f}}=0\\
\implies \vec{M}_T\mathit{f}=\vec{0}
\implies \mathit{f}=0 \because \vec{M}_T \ne \vec{0}\\
\implies N\brak{T}=\cbrak{0}
\end{align}}
\\ [0.5ex] 
\hline
\end{tabular}
\caption{Proof for Non-Singular and linear transformation T}
\label{table:2}
\end{center}
\vspace{-0.5cm}
\end{table*}

\renewcommand{\thetable}{3}
\begin{table*}[ht!]
\begin{center}
\begin{tabular}{|p{4.7cm}|p{10cm}|}
\hline
\textbf{Statement} & \textbf{Derivation} \\[0.5ex]
\hline
T is not invertible & As $\vec{M}_T$ is a non-square matrix with dimensions $\brak{n+2}\times\brak{n+1}$, the transformation T is not invertible
\\ [0.5ex] 
\hline
$\vec{M}_D$ is left inverse of $\vec{M}_T$ 
& \parbox{10cm}{\begin{align}
\vec{M}_D\vec{M}_T=\vec{I}_{n+1}\\
\implies \vec{M}_D=\myvec{0&1&0&0&\cdots&0\\0&0&2&0&\cdots&0\\0&0&0&3&\cdots&0\\\vdots&\vdots&\vdots&\vdots&\cdots&\vdots\\0&0&0&0&\cdots&n+1}
\end{align}}
\\ [0.5ex] 
\hline
\end{tabular}
\caption{Non-Invertibility of transformation T}
\label{table:3}
\end{center}
\vspace{-0.5cm}
\end{table*}

\end{document}