\documentclass[journal,12pt,twocolumn]{IEEEtran}

\usepackage{setspace}
\usepackage{gensymb}

\singlespacing
\usepackage[cmex10]{amsmath}

\usepackage{amsthm}

\usepackage{mathrsfs}
\usepackage{txfonts}
\usepackage{stfloats}
\usepackage{bm}
\usepackage{cite}
\usepackage{cases}
\usepackage{subfig}
\usepackage{float}
\usepackage{longtable}
\usepackage{multirow}
\usepackage{caption}
%\usepackage[font=bf,labelfont=bf]{caption}

\usepackage{enumitem}
\usepackage{mathtools}
\usepackage{steinmetz}
\usepackage{tikz}
\usepackage{circuitikz}
\usepackage{verbatim}
\usepackage{tfrupee}
\usepackage[breaklinks=true]{hyperref}

\usepackage{tkz-euclide}

\usetikzlibrary{calc,math}
\usepackage{listings}
    \usepackage{color}                                            %%
    \usepackage{array}                                            %%
    \usepackage{longtable}                                        %%
    \usepackage{calc}                                             %%
    \usepackage{multirow}                                         %%
    \usepackage{hhline}                                           %%
    \usepackage{ifthen}                                           %%
    \usepackage{lscape}     
\usepackage{multicol}
\usepackage{chngcntr}


\DeclareMathOperator*{\Res}{Res}

\renewcommand\thesection{\arabic{section}}
\renewcommand\thesubsection{\thesection.\arabic{subsection}}
\renewcommand\thesubsubsection{\thesubsection.\arabic{subsubsection}}

\renewcommand\thesectiondis{\arabic{section}}
\renewcommand\thesubsectiondis{\thesectiondis.\arabic{subsection}}
\renewcommand\thesubsubsectiondis{\thesubsectiondis.\arabic{subsubsection}}
\numberwithin{table}{section}

\hyphenation{op-tical net-works semi-conduc-tor}
\def\inputGnumericTable{}                                 %%

\lstset{
%language=C,
frame=single, 
breaklines=true,
columns=fullflexible
}
\makeatletter
\renewcommand*\env@matrix[1][*\c@MaxMatrixCols c]{%
  \hskip -\arraycolsep
  \let\@ifnextchar\new@ifnextchar
  \array{#1}}
\makeatother
\begin{document}


\newtheorem{theorem}{Theorem}[section]
\newtheorem{problem}{Problem}
\newtheorem{proposition}{Proposition}[section]
\newtheorem{lemma}{Lemma}[section]
\newtheorem{corollary}[theorem]{Corollary}
\newtheorem{example}{Example}[section]
\newtheorem{definition}[problem]{Definition}

\newcommand{\BEQA}{\begin{eqnarray}}
\newcommand{\EEQA}{\end{eqnarray}}
\newcommand{\define}{\stackrel{\triangle}{=}}
\bibliographystyle{IEEEtran}
\providecommand{\mbf}{\mathbf}
\providecommand{\pr}[1]{\ensuremath{\Pr\left(#1\right)}}
\providecommand{\qfunc}[1]{\ensuremath{Q\left(#1\right)}}
\providecommand{\sbrak}[1]{\ensuremath{{}\left[#1\right]}}
\providecommand{\lsbrak}[1]{\ensuremath{{}\left[#1\right.}}
\providecommand{\rsbrak}[1]{\ensuremath{{}\left.#1\right]}}
\providecommand{\brak}[1]{\ensuremath{\left(#1\right)}}
\providecommand{\lbrak}[1]{\ensuremath{\left(#1\right.}}
\providecommand{\rbrak}[1]{\ensuremath{\left.#1\right)}}
\providecommand{\cbrak}[1]{\ensuremath{\left\{#1\right\}}}
\providecommand{\lcbrak}[1]{\ensuremath{\left\{#1\right.}}
\providecommand{\rcbrak}[1]{\ensuremath{\left.#1\right\}}}
\theoremstyle{remark}
\newtheorem{rem}{Remark}
\newcommand{\sgn}{\mathop{\mathrm{sgn}}}
\providecommand{\abs}[1]{\left\vert#1\right\vert}
\providecommand{\res}[1]{\Res\displaylimits_{#1}} 
\providecommand{\norm}[1]{\left\lVert#1\right\rVert}
%\providecommand{\norm}[1]{\lVert#1\rVert}
\providecommand{\mtx}[1]{\mathbf{#1}}
\providecommand{\mean}[1]{E\left[ #1 \right]}
\providecommand{\fourier}{\overset{\mathcal{F}}{ \rightleftharpoons}}
%\providecommand{\hilbert}{\overset{\mathcal{H}}{ \rightleftharpoons}}
\providecommand{\system}{\overset{\mathcal{H}}{ \longleftrightarrow}}
	%\newcommand{\solution}[2]{\textbf{Solution:}{#1}}
\newcommand{\solution}{\noindent \textbf{Solution: }}
\newcommand{\cosec}{\,\text{cosec}\,}
\providecommand{\dec}[2]{\ensuremath{\overset{#1}{\underset{#2}{\gtrless}}}}
\newcommand{\myvec}[1]{\ensuremath{\begin{pmatrix}#1\end{pmatrix}}}
\newcommand{\mydet}[1]{\ensuremath{\begin{vmatrix}#1\end{vmatrix}}}
\numberwithin{equation}{subsection}
\makeatletter
\@addtoreset{figure}{problem}
\makeatother
\let\StandardTheFigure\thefigure
\let\vec\mathbf
\renewcommand{\thefigure}{\theproblem}
\def\putbox#1#2#3{\makebox[0in][l]{\makebox[#1][l]{}\raisebox{\baselineskip}[0in][0in]{\raisebox{#2}[0in][0in]{#3}}}}
     \def\rightbox#1{\makebox[0in][r]{#1}}
     \def\centbox#1{\makebox[0in]{#1}}
     \def\topbox#1{\raisebox{-\baselineskip}[0in][0in]{#1}}
     \def\midbox#1{\raisebox{-0.5\baselineskip}[0in][0in]{#1}}
\vspace{3cm}
\title{Assignment 13}
\author{Yenigalla Samyuktha\\EE20MTECH14019}
\maketitle
\newpage
\bigskip
\renewcommand{\thefigure}{\theenumi}
\renewcommand{\thetable}{\theenumi}
\begin{abstract}
This document explains a proof in linear transformations.
\end{abstract}
Download all latex-tikz codes from 
%
\begin{lstlisting}
https://github.com/EE20MTECH14019/EE5609/tree/master/Assignment_13
\end{lstlisting}
%
\section{Problem}
Let $\vec{A}$ be an $m\times n$ matrix with real entries. Prove that $\vec{A}=\vec{0}$ is and only if $Trace\brak{\vec{A}^T\vec{A}}=0$.
\section{Solution}
\begin{enumerate}
\item Given $Trace\brak{\vec{A}^T\vec{A}}=0$, to prove $\vec{A}=\vec{0}$:
Using Singular value decomposition of any $m\times n$ matrix with real entries, we can write
\begin{align}
\vec{A}=\vec{U}\vec{\Sigma}\vec{V}^T
\end{align}
Now,
\begin{align}
Trace\brak{\vec{A}^T\vec{A}}\label{eq1}\\
=Trace\brak{\brak{\vec{U}\vec{\Sigma}\vec{V}^T}^T\brak{\vec{U}\vec{\Sigma}\vec{V}^T}}\\
=Trace\brak{\vec{V}\vec{\Sigma}^T\vec{U}^T\vec{U}\vec{\Sigma}\vec{V}^T}\label{eq2}
\end{align}
As $\vec{U}$ and $\vec{V}$ are unitary matrices, $\vec{V}^T\vec{V}=\vec{I}$, $\vec{U}^T\vec{U}=\vec{I}$ and $\vec{\Sigma}$ is a diagonal matrix, we can re-write equation \eqref{eq2} as,
\begin{align}
Trace\brak{\vec{V}\vec{\Sigma}^2\vec{V}^T}\label{eq3}
\end{align}
From the Cyclic property of trace of product of matrices, \eqref{eq3} becomes
\begin{align}
Trace\brak{\vec{V}^T\vec{V}\vec{\Sigma}^2}\\
=\sum\limits_{i=1}^r \sigma_{i}^2\label{eq4}
\end{align}
where $\sigma_{i}$'s are singular values of matrix $\vec{A}$ and r is the rank of matrix $\vec{A}$. Considering $trace\brak{\vec{A}^T\vec{A}}=0$, from \eqref{eq4}
\begin{align}
\sum\limits_{i=1}^r \sigma_{i}^2=0\\
\implies \sigma_{i}^2=0 \; \forall i=1,2,\dots r \label{eq6}
\end{align}
As $\vec{A}$ has real entries, from \eqref{eq6}
\begin{align}
\implies \sigma_i=0 \; \forall i=1,2,\dots r \label{eq7}
\end{align}
\begin{align}
\because Rank\brak{\vec{A}}=\# \text{non-zero singular values}\label{eq8}\\
=\# \text{non-zero diagonal elements of }\vec{\Sigma}
\end{align}
From \eqref{eq7} and \eqref{eq8},
\begin{align}
Rank\brak{\vec{A}}=0\\
\implies \vec{A}=\vec{0}
\end{align}
\item Given $\vec{A}=\vec{0}$, to prove $Trace\brak{\vec{A}^T\vec{A}}=0$:
\begin{align}
\vec{A}=0\\
\implies \vec{A}^T\vec{A}=\vec{0}\\
\implies Trace\brak{\vec{A}^T\vec{A}}=0
\end{align}
\end{enumerate}
\end{document}