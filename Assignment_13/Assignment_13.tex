\documentclass[journal,12pt,twocolumn]{IEEEtran}

\usepackage{setspace}
\usepackage{gensymb}

\singlespacing
\usepackage[cmex10]{amsmath}

\usepackage{amsthm}
\usepackage{mathrsfs}
\usepackage{txfonts}
\usepackage{stfloats}
\usepackage{bm}
\usepackage{cite}
\usepackage{cases}
\usepackage{subfig}
\usepackage{float}
\usepackage{longtable}
\usepackage{multirow}
\usepackage{caption}
%\usepackage[font=bf,labelfont=bf]{caption}

\usepackage{enumitem}
\usepackage{mathtools}
\usepackage{steinmetz}
\usepackage{tikz}
\usepackage{circuitikz}
\usepackage{verbatim}
\usepackage{tfrupee}
\usepackage[breaklinks=true]{hyperref}

\usepackage{tkz-euclide}

\usetikzlibrary{calc,math}
\usepackage{listings}
    \usepackage{color}                                            %%
    \usepackage{array}                                            %%
    \usepackage{longtable}                                        %%
    \usepackage{calc}                                             %%
    \usepackage{multirow}                                         %%
    \usepackage{hhline}                                           %%
    \usepackage{ifthen}                                           %%
    \usepackage{lscape}     
\usepackage{multicol}
\usepackage{chngcntr}


\DeclareMathOperator*{\Res}{Res}

\renewcommand\thesection{\arabic{section}}
\renewcommand\thesubsection{\thesection.\arabic{subsection}}
\renewcommand\thesubsubsection{\thesubsection.\arabic{subsubsection}}

\renewcommand\thesectiondis{\arabic{section}}
\renewcommand\thesubsectiondis{\thesectiondis.\arabic{subsection}}
\renewcommand\thesubsubsectiondis{\thesubsectiondis.\arabic{subsubsection}}
\numberwithin{table}{section}

\hyphenation{op-tical net-works semi-conduc-tor}
\def\inputGnumericTable{}                                 %%

\lstset{
%language=C,
frame=single, 
breaklines=true,
columns=fullflexible
}
\makeatletter
\renewcommand*\env@matrix[1][*\c@MaxMatrixCols c]{%
  \hskip -\arraycolsep
  \let\@ifnextchar\new@ifnextchar
  \array{#1}}
\makeatother
\begin{document}


\newtheorem{theorem}{Theorem}[section]
\newtheorem{problem}{Problem}
\newtheorem{proposition}{Proposition}[section]
\newtheorem{lemma}{Lemma}[section]
\newtheorem{corollary}[theorem]{Corollary}
\newtheorem{example}{Example}[section]
\newtheorem{definition}[problem]{Definition}

\newcommand{\BEQA}{\begin{eqnarray}}
\newcommand{\EEQA}{\end{eqnarray}}
\newcommand{\define}{\stackrel{\triangle}{=}}
\bibliographystyle{IEEEtran}
\providecommand{\mbf}{\mathbf}
\providecommand{\pr}[1]{\ensuremath{\Pr\left(#1\right)}}
\providecommand{\qfunc}[1]{\ensuremath{Q\left(#1\right)}}
\providecommand{\sbrak}[1]{\ensuremath{{}\left[#1\right]}}
\providecommand{\lsbrak}[1]{\ensuremath{{}\left[#1\right.}}
\providecommand{\rsbrak}[1]{\ensuremath{{}\left.#1\right]}}
\providecommand{\brak}[1]{\ensuremath{\left(#1\right)}}
\providecommand{\lbrak}[1]{\ensuremath{\left(#1\right.}}
\providecommand{\rbrak}[1]{\ensuremath{\left.#1\right)}}
\providecommand{\cbrak}[1]{\ensuremath{\left\{#1\right\}}}
\providecommand{\lcbrak}[1]{\ensuremath{\left\{#1\right.}}
\providecommand{\rcbrak}[1]{\ensuremath{\left.#1\right\}}}
\theoremstyle{remark}
\newtheorem{rem}{Remark}
\newcommand{\sgn}{\mathop{\mathrm{sgn}}}
\providecommand{\abs}[1]{\left\vert#1\right\vert}
\providecommand{\res}[1]{\Res\displaylimits_{#1}} 
\providecommand{\norm}[1]{\left\lVert#1\right\rVert}
%\providecommand{\norm}[1]{\lVert#1\rVert}
\providecommand{\mtx}[1]{\mathbf{#1}}
\providecommand{\mean}[1]{E\left[ #1 \right]}
\providecommand{\fourier}{\overset{\mathcal{F}}{ \rightleftharpoons}}
%\providecommand{\hilbert}{\overset{\mathcal{H}}{ \rightleftharpoons}}
\providecommand{\system}{\overset{\mathcal{H}}{ \longleftrightarrow}}
	%\newcommand{\solution}[2]{\textbf{Solution:}{#1}}
\newcommand{\solution}{\noindent \textbf{Solution: }}
\newcommand{\cosec}{\,\text{cosec}\,}
\providecommand{\dec}[2]{\ensuremath{\overset{#1}{\underset{#2}{\gtrless}}}}
\newcommand{\myvec}[1]{\ensuremath{\begin{pmatrix}#1\end{pmatrix}}}
\newcommand{\mydet}[1]{\ensuremath{\begin{vmatrix}#1\end{vmatrix}}}
\numberwithin{equation}{subsection}
\makeatletter
\@addtoreset{figure}{problem}
\makeatother
\let\StandardTheFigure\thefigure
\let\vec\mathbf
\renewcommand{\thefigure}{\theproblem}
\def\putbox#1#2#3{\makebox[0in][l]{\makebox[#1][l]{}\raisebox{\baselineskip}[0in][0in]{\raisebox{#2}[0in][0in]{#3}}}}
     \def\rightbox#1{\makebox[0in][r]{#1}}
     \def\centbox#1{\makebox[0in]{#1}}
     \def\topbox#1{\raisebox{-\baselineskip}[0in][0in]{#1}}
     \def\midbox#1{\raisebox{-0.5\baselineskip}[0in][0in]{#1}}
\vspace{3cm}
\title{Assignment 13}
\author{Yenigalla Samyuktha\\EE20MTECH14019}
\maketitle
\newpage
\bigskip
\renewcommand{\thefigure}{\theenumi}
\renewcommand{\thetable}{1}
\setlength{\tabcolsep}{20pt}
\renewcommand{\arraystretch}{1.5}
\begin{abstract}
This document explains a proof in linear transformations.
\end{abstract}
Download all latex-tikz codes from 
%
\begin{lstlisting}
https://github.com/EE20MTECH14019/EE5609/tree/master/Assignment_13
\end{lstlisting}
%
\section{Problem}
Let $\vec{A}$ be an $m\times n$ matrix with real entries. Prove that $\vec{A}=\vec{0}$ is and only if $tr\brak{\vec{A}^T\vec{A}}=0$.
\section{Solution}
The proof is given in the table \ref{table:2} and the properties used for the proof are listed in the table \ref{table:1}
\renewcommand{\thetable}{1}
\begin{table*}[ht!]
\begin{center}
\begin{tabular}{|l|l|}
\hline
\multicolumn{2}{|c|}{
\textbf{Properties Used}}\\
\hline
SVD of matrix $\vec{A}$ & $\vec{A}=\vec{U}\vec{\Sigma}\vec{V}^T$\\[0.5ex]
\hline
$\vec{U}$ is unitary & $\vec{U}^T\vec{U}=\vec{I}$\\[0.5ex] 
\hline
$\vec{V}$ is unitary & $\vec{V}^T\vec{V}=\vec{I}$\\[0.5ex] 
\hline
$\vec{\Sigma}$ is diagonal & $\vec{\Sigma}^T\vec{\Sigma}=\vec{\Sigma}^2$\\[0.5ex] 
\hline
Cyclic property & $tr\brak{\vec{ABC}}=tr\brak{\vec{CAB}}$\\[0.5ex] 
\hline
$Rank\brak{\vec{A}}$ & $Rank\brak{\vec{A}}=\# \text{non-zero singular values}$\\[0.5ex] 
\hline
\end{tabular}
\caption{Properties}
\label{table:1}
\end{center}
\vspace{-0.5cm}
\end{table*}
\renewcommand{\thetable}{2}
\begin{table*}[ht!]
\begin{center}
\begin{tabular}{|l|l|}
\hline
\textbf{Statement} & \textbf{Proof} \\[0.5ex]
\hline
$tr\brak{\vec{A}^T\vec{A}}=0 \implies \vec{A}=\vec{0}$ & 
 \parbox{10cm}{\begin{align}
    tr\brak{\vec{A}^T\vec{A}}=tr\brak{\vec{V}\vec{\Sigma}^T\vec{U}^T\vec{U}\vec{\Sigma}\vec{V}^T}\\
	=tr\brak{\vec{V}\vec{\Sigma}^T\vec{\Sigma}\vec{V}^T}\\
	=tr\brak{\vec{V}\vec{\Sigma}^2\vec{V}^T}\\
	=tr\brak{\vec{V}^T\vec{V}\vec{\Sigma}^2}\\
	=tr\brak{\vec{\Sigma}^2}\\
	=\sum\limits_{i=1}^r \sigma_{i}^2 \\ 
	 tr\brak{\vec{A}^T\vec{A}}=0 \\
	\implies \sum\limits_{i=1}^r \sigma_{i}^2=0 \; \forall i=1,2,\dots r\\
	\implies \sigma_i=0 \; \forall i=1,2,\dots r \\
	\therefore Rank\brak{\vec{A}}=\# \text{non-zero singular values}=0\\
	\implies \vec{A}=\vec{0}
\end{align}}
\\ [0.5ex] 
\hline
$\vec{A}=\vec{0} \implies  tr\brak{\vec{A}^T\vec{A}}=0$
& \parbox{10cm}{\begin{align}
    \vec{A}=\vec{0}\\
    \implies \vec{A}^T\vec{A}=\vec{0}\\
    \implies tr\brak{\vec{A}^T\vec{A}}=0
\end{align}}
\\ [0.5ex] 
\hline
\end{tabular}
\caption{Proofs}
\label{table:2}
\end{center}
\vspace{-0.5cm}
\end{table*}
\end{document}