\documentclass[journal,12pt,twocolumn]{IEEEtran}

\usepackage{setspace}
\usepackage{gensymb}

\singlespacing


\usepackage[cmex10]{amsmath}

\usepackage{amsthm}

\usepackage{mathrsfs}
\usepackage{txfonts}
\usepackage{stfloats}
\usepackage{bm}
\usepackage{cite}
\usepackage{cases}
\usepackage{subfig}

\usepackage{longtable}
\usepackage{multirow}

\usepackage{enumitem}
\usepackage{mathtools}
\usepackage{steinmetz}
\usepackage{tikz}
\usepackage{circuitikz}
\usepackage{verbatim}
\usepackage{tfrupee}
\usepackage[breaklinks=true]{hyperref}

\usepackage{tkz-euclide}

\usetikzlibrary{calc,math}
\usepackage{listings}
    \usepackage{color}                                            %%
    \usepackage{array}                                            %%
    \usepackage{longtable}                                        %%
    \usepackage{calc}                                             %%
    \usepackage{multirow}                                         %%
    \usepackage{hhline}                                           %%
    \usepackage{ifthen}                                           %%
    \usepackage{lscape}     
\usepackage{multicol}
\usepackage{chngcntr}

\DeclareMathOperator*{\Res}{Res}

\renewcommand\thesection{\arabic{section}}
\renewcommand\thesubsection{\thesection.\arabic{subsection}}
\renewcommand\thesubsubsection{\thesubsection.\arabic{subsubsection}}

\renewcommand\thesectiondis{\arabic{section}}
\renewcommand\thesubsectiondis{\thesectiondis.\arabic{subsection}}
\renewcommand\thesubsubsectiondis{\thesubsectiondis.\arabic{subsubsection}}


\hyphenation{op-tical net-works semi-conduc-tor}
\def\inputGnumericTable{}                                 %%

\lstset{
%language=C,
frame=single, 
breaklines=true,
columns=fullflexible
}
\begin{document}


\newtheorem{theorem}{Theorem}[section]
\newtheorem{problem}{Problem}
\newtheorem{proposition}{Proposition}[section]
\newtheorem{lemma}{Lemma}[section]
\newtheorem{corollary}[theorem]{Corollary}
\newtheorem{example}{Example}[section]
\newtheorem{definition}[problem]{Definition}

\newcommand{\BEQA}{\begin{eqnarray}}
\newcommand{\EEQA}{\end{eqnarray}}
\newcommand{\define}{\stackrel{\triangle}{=}}
\bibliographystyle{IEEEtran}
\providecommand{\mbf}{\mathbf}
\providecommand{\pr}[1]{\ensuremath{\Pr\left(#1\right)}}
\providecommand{\qfunc}[1]{\ensuremath{Q\left(#1\right)}}
\providecommand{\sbrak}[1]{\ensuremath{{}\left[#1\right]}}
\providecommand{\lsbrak}[1]{\ensuremath{{}\left[#1\right.}}
\providecommand{\rsbrak}[1]{\ensuremath{{}\left.#1\right]}}
\providecommand{\brak}[1]{\ensuremath{\left(#1\right)}}
\providecommand{\lbrak}[1]{\ensuremath{\left(#1\right.}}
\providecommand{\rbrak}[1]{\ensuremath{\left.#1\right)}}
\providecommand{\cbrak}[1]{\ensuremath{\left\{#1\right\}}}
\providecommand{\lcbrak}[1]{\ensuremath{\left\{#1\right.}}
\providecommand{\rcbrak}[1]{\ensuremath{\left.#1\right\}}}
\theoremstyle{remark}
\newtheorem{rem}{Remark}
\newcommand{\sgn}{\mathop{\mathrm{sgn}}}
\providecommand{\abs}[1]{\left\vert#1\right\vert}
\providecommand{\res}[1]{\Res\displaylimits_{#1}} 
\providecommand{\norm}[1]{\left\lVert#1\right\rVert}
%\providecommand{\norm}[1]{\lVert#1\rVert}
\providecommand{\mtx}[1]{\mathbf{#1}}
\providecommand{\mean}[1]{E\left[ #1 \right]}
\providecommand{\fourier}{\overset{\mathcal{F}}{ \rightleftharpoons}}
%\providecommand{\hilbert}{\overset{\mathcal{H}}{ \rightleftharpoons}}
\providecommand{\system}{\overset{\mathcal{H}}{ \longleftrightarrow}}
	%\newcommand{\solution}[2]{\textbf{Solution:}{#1}}
\newcommand{\solution}{\noindent \textbf{Solution: }}
\newcommand{\cosec}{\,\text{cosec}\,}
\providecommand{\dec}[2]{\ensuremath{\overset{#1}{\underset{#2}{\gtrless}}}}
\newcommand{\myvec}[1]{\ensuremath{\begin{pmatrix}#1\end{pmatrix}}}
\newcommand{\mydet}[1]{\ensuremath{\begin{vmatrix}#1\end{vmatrix}}}
\numberwithin{equation}{subsection}
\makeatletter
\@addtoreset{figure}{problem}
\makeatother
\let\StandardTheFigure\thefigure
\let\vec\mathbf
\renewcommand{\thefigure}{\theproblem}
\def\putbox#1#2#3{\makebox[0in][l]{\makebox[#1][l]{}\raisebox{\baselineskip}[0in][0in]{\raisebox{#2}[0in][0in]{#3}}}}
     \def\rightbox#1{\makebox[0in][r]{#1}}
     \def\centbox#1{\makebox[0in]{#1}}
     \def\topbox#1{\raisebox{-\baselineskip}[0in][0in]{#1}}
     \def\midbox#1{\raisebox{-0.5\baselineskip}[0in][0in]{#1}}
\vspace{3cm}
\title{Assignment 9}
\author{Yenigalla Samyuktha\\EE20MTECH14019}
\maketitle
\newpage
\bigskip
\renewcommand{\thefigure}{\theenumi}
\renewcommand{\thetable}{\theenumi}
\begin{abstract}
This document checks for the vectors in the subspace spanned by given vectors.
\end{abstract}
Download all latex-tikz codes from 
%
\begin{lstlisting}
https://github.com/EE20MTECH14019/EE5609/tree/master/Assignment_9
\end{lstlisting}
%
\section{Problem}
Let
\begin{align} 
\alpha_1=\myvec{1&1&-2&1}^T \\
\alpha_2=\myvec{3&0&4&-1}^T \\
\alpha_3=\myvec{-1&2&5&2}^T
\end{align}
Let
\begin{align}
\alpha=\myvec{4&-5&9&-7}^T \\
\beta=\myvec{3&1&-4&4}^T \\
\gamma=\myvec{-1&1&0&1}^T
\end{align}
\begin{enumerate}
\item Which of the vectors $\alpha$, $\beta$, $\gamma$ are in the subspace of $\mathbb{R}^4$ spanned by $\alpha_i$?
\item Which of the vectors $\alpha$, $\beta$, $\gamma$ are in the subspace of $\mathbb{C}^4$ spanned by $\alpha_i$?
\item Does this suggest a theorem?
\end{enumerate}
\section{Solution}
\begin{enumerate}
\item The linear combination of $\alpha_i$ for $i=1,2,3$ spans subspace S. We can write,
\begin{align}
c_1\myvec{1\\1\\-2\\1}+c_2\myvec{3\\0\\4\\-1}+c_3\myvec{-1\\2\\5\\2}=\text{span(S)}
\end{align}
where $c_1$,$c_2$,$c_3$ are scalars.
Vectors in matrix form is given by
\begin{align}
\vec{A}=\myvec{1&3&-1\\1&0&2\\-2&4&5\\1&-1&2}
\end{align}
We can observe that the columns of matrix $\vec{A}$ formed by vectors $\alpha_i$ are independent as the rank of matrix is 3. Hence $\alpha_i$ forms basis for subspace S.
\begin{enumerate}
\item \textbf{Checking for $\alpha$}:
To check if a solution exists for $\vec{AX}=\alpha$. The corresponding agumented matrix can be written as,
\begin{align} \label{eq:auga}
\myvec{\vec{A}&\alpha}=\myvec{1&3&-1&4\\1&0&2&-5\\-2&4&5&9\\1&-1&2&-7}
\end{align}
On performing row-reduction on \eqref{eq:auga}, 
\begin{align}\label{eq:refa}
\myvec{\vec{A}&\alpha}=\myvec{1&0&0&-3\\0&1&0&2\\0&0&1&-1\\0&0&0&0}
\end{align}
As Rank($\myvec{\vec{A}&\alpha}$)=Rank($\vec{A}$)=3, the vector $\alpha$ can be represented as linear combination of $\alpha_i$. From equation \eqref{eq:refa}, we can write
\begin{align}
-3\myvec{1\\1\\-2\\1}+2\myvec{3\\0\\4\\-1}-1\myvec{-1\\2\\5\\2}=\myvec{4\\-5\\9\\-7}
\end{align}
Hence $\alpha$ is in the subspace S.

\item \textbf{Checking for $\beta$}:
To check if a solution exists for $\vec{AX}=\beta$. The corresponding agumented matrix can be written as,
\begin{align} \label{eq:augb}
\myvec{\vec{A}&\beta}=\myvec{1&3&-1&3\\1&0&2&1\\-2&4&5&-4\\1&-1&2&4}
\end{align}
On performing row-reduction on \eqref{eq:augb}, 
\begin{align}\label{eq:refb}
\myvec{\vec{A}&\beta}=\myvec{1&0&0&0\\0&1&0&0\\0&0&1&0\\0&0&0&1}
\end{align}
As Rank($\myvec{\vec{A}&\beta}$)=4 and Rank($\vec{A}$)=3, Solution doesn't exist for $AX=\beta$ and hence $\beta$ is not in the subspace S.

\item \textbf{Checking for $\gamma$}:
To check if a solution exists for $\vec{AX}=\gamma$. The corresponding agumented matrix can be written as,
\begin{align} \label{eq:augg}
\myvec{\vec{A}&\gamma}=\myvec{1&3&-1&-1\\1&0&2&1\\-2&4&5&0\\1&-1&2&1}
\end{align}
On performing row-reduction on \eqref{eq:augg}, 
\begin{align}\label{eq:refg}
\myvec{\vec{A}&\gamma}=\myvec{1&0&0&0\\0&1&0&0\\0&0&1&0\\0&0&0&1}
\end{align}
As Rank($\myvec{\vec{A}&\gamma}$)=4 and Rank($\vec{A}$)=3, Solution doesn't exist for $AX=\gamma$ and hence $\gamma$ is not in the subspace S.
\end{enumerate}
\item In part 1, we haven't considered the field to be either $\mathbb{R}$ or $\mathbb{C}$. The above equations solved holds for field $\mathbb{C}$ and that implies, they hold for field $\mathbb{R}$ also. Hence $\alpha$ is in the subspace and $\beta$ and $\gamma$ are not in the subspace.
\item \textbf{Theorem suggested:}
Let $\mathbb{F}_1$ and $\mathbb{F}_2$ are two fields where $\mathbb{F}_2$ is subfield of $\mathbb{F}_1$. Let $\alpha_i$, i=1,2,3...,n forms basis for subspace of $\mathbb{F}_2^n$ and a vector $\alpha \in \mathbb{F}_2^n$. Then $\alpha$ is in the subspace of $\mathbb{F}_2^n$ spanned by $\alpha_i$, i=1,2,3...,n if only if $\alpha$ is in the subspace of $\mathbb{F}_1^n$ spanned by $\alpha_i$, i=1,2,3...,n.
\end{enumerate}
\end{document}