\documentclass[journal,12pt,twocolumn]{IEEEtran}

\usepackage{setspace}
\usepackage{gensymb}

\singlespacing


\usepackage[cmex10]{amsmath}

\usepackage{amsthm}

\usepackage{mathrsfs}
\usepackage{txfonts}
\usepackage{stfloats}
\usepackage{bm}
\usepackage{cite}
\usepackage{cases}
\usepackage{subfig}

\usepackage{longtable}
\usepackage{multirow}

\usepackage{enumitem}
\usepackage{mathtools}
\usepackage{steinmetz}
\usepackage{tikz}
\usepackage{circuitikz}
\usepackage{verbatim}
\usepackage{tfrupee}
\usepackage[breaklinks=true]{hyperref}

\usepackage{tkz-euclide}

\usetikzlibrary{calc,math}
\usepackage{listings}
    \usepackage{color}                                            %%
    \usepackage{array}                                            %%
    \usepackage{longtable}                                        %%
    \usepackage{calc}                                             %%
    \usepackage{multirow}                                         %%
    \usepackage{hhline}                                           %%
    \usepackage{ifthen}                                           %%
    \usepackage{lscape}     
\usepackage{multicol}
\usepackage{chngcntr}

\DeclareMathOperator*{\Res}{Res}

\renewcommand\thesection{\arabic{section}}
\renewcommand\thesubsection{\thesection.\arabic{subsection}}
\renewcommand\thesubsubsection{\thesubsection.\arabic{subsubsection}}

\renewcommand\thesectiondis{\arabic{section}}
\renewcommand\thesubsectiondis{\thesectiondis.\arabic{subsection}}
\renewcommand\thesubsubsectiondis{\thesubsectiondis.\arabic{subsubsection}}


\hyphenation{op-tical net-works semi-conduc-tor}
\def\inputGnumericTable{}                                 %%

\lstset{
%language=C,
frame=single, 
breaklines=true,
columns=fullflexible
}
\begin{document}


\newtheorem{theorem}{Theorem}[section]
\newtheorem{problem}{Problem}
\newtheorem{proposition}{Proposition}[section]
\newtheorem{lemma}{Lemma}[section]
\newtheorem{corollary}[theorem]{Corollary}
\newtheorem{example}{Example}[section]
\newtheorem{definition}[problem]{Definition}

\newcommand{\BEQA}{\begin{eqnarray}}
\newcommand{\EEQA}{\end{eqnarray}}
\newcommand{\define}{\stackrel{\triangle}{=}}
\bibliographystyle{IEEEtran}
\providecommand{\mbf}{\mathbf}
\providecommand{\pr}[1]{\ensuremath{\Pr\left(#1\right)}}
\providecommand{\qfunc}[1]{\ensuremath{Q\left(#1\right)}}
\providecommand{\sbrak}[1]{\ensuremath{{}\left[#1\right]}}
\providecommand{\lsbrak}[1]{\ensuremath{{}\left[#1\right.}}
\providecommand{\rsbrak}[1]{\ensuremath{{}\left.#1\right]}}
\providecommand{\brak}[1]{\ensuremath{\left(#1\right)}}
\providecommand{\lbrak}[1]{\ensuremath{\left(#1\right.}}
\providecommand{\rbrak}[1]{\ensuremath{\left.#1\right)}}
\providecommand{\cbrak}[1]{\ensuremath{\left\{#1\right\}}}
\providecommand{\lcbrak}[1]{\ensuremath{\left\{#1\right.}}
\providecommand{\rcbrak}[1]{\ensuremath{\left.#1\right\}}}
\theoremstyle{remark}
\newtheorem{rem}{Remark}
\newcommand{\sgn}{\mathop{\mathrm{sgn}}}
\providecommand{\abs}[1]{\left\vert#1\right\vert}
\providecommand{\res}[1]{\Res\displaylimits_{#1}} 
\providecommand{\norm}[1]{\left\lVert#1\right\rVert}
%\providecommand{\norm}[1]{\lVert#1\rVert}
\providecommand{\mtx}[1]{\mathbf{#1}}
\providecommand{\mean}[1]{E\left[ #1 \right]}
\providecommand{\fourier}{\overset{\mathcal{F}}{ \rightleftharpoons}}
%\providecommand{\hilbert}{\overset{\mathcal{H}}{ \rightleftharpoons}}
\providecommand{\system}{\overset{\mathcal{H}}{ \longleftrightarrow}}
	%\newcommand{\solution}[2]{\textbf{Solution:}{#1}}
\newcommand{\solution}{\noindent \textbf{Solution: }}
\newcommand{\cosec}{\,\text{cosec}\,}
\providecommand{\dec}[2]{\ensuremath{\overset{#1}{\underset{#2}{\gtrless}}}}
\newcommand{\myvec}[1]{\ensuremath{\begin{pmatrix}#1\end{pmatrix}}}
\newcommand{\mydet}[1]{\ensuremath{\begin{vmatrix}#1\end{vmatrix}}}
\numberwithin{equation}{subsection}
\makeatletter
\@addtoreset{figure}{problem}
\makeatother
\let\StandardTheFigure\thefigure
\let\vec\mathbf
\renewcommand{\thefigure}{\theproblem}
\def\putbox#1#2#3{\makebox[0in][l]{\makebox[#1][l]{}\raisebox{\baselineskip}[0in][0in]{\raisebox{#2}[0in][0in]{#3}}}}
     \def\rightbox#1{\makebox[0in][r]{#1}}
     \def\centbox#1{\makebox[0in]{#1}}
     \def\topbox#1{\raisebox{-\baselineskip}[0in][0in]{#1}}
     \def\midbox#1{\raisebox{-0.5\baselineskip}[0in][0in]{#1}}
\vspace{3cm}
\title{Assignment 10}
\author{Yenigalla Samyuktha\\EE20MTECH14019}
\maketitle
\newpage
\bigskip
\renewcommand{\thefigure}{\theenumi}
\renewcommand{\thetable}{\theenumi}
\begin{abstract}
This document explains a proof on linear transformations.
\end{abstract}
Download all latex-tikz codes from 
%
\begin{lstlisting}
https://github.com/EE20MTECH14019/EE5609/tree/master/Assignment_10
\end{lstlisting}
%
\section{Problem}
Let T be a linear transformation from $\mathbb{R}^3$ into $\mathbb{R}^2$, and let U be a linear transformation from $\mathbb{R}^2$ into $\mathbb{R}^3$. Prove that the transformation UT is not invertible. Generalize the theorem.
\section{Solution}
We have two transformations 
\begin{align}
T:\mathbb{R}^3\rightarrow\mathbb{R}^2\\
U:\mathbb{R}^2\rightarrow\mathbb{R}^3
\end{align}
Let $\vec{v},\vec{x}\in\mathbb{R}^3$ and $\vec{w}\in\mathbb{R}^2$. Hence we can write the transformations in matrix form as,
\begin{align}
T(\vec{v})=\vec{A}\vec{v} \label{Tmat}\\
U(\vec{w})=\vec{B}\vec{w} \label{Umat}
\end{align}
Where transformation matrix $\vec{A}$ has dimension of 2x3. Hence we can say,
\begin{align}\label{eqA}
Max(Rank(\vec{A}))=2
\end{align}
And transformation matrix $\vec{B}$ has dimension of 3x2. Hence we can say,
\begin{align}\label{eqB}
Max(Rank(\vec{B}))=2
\end{align}
Now we define the transformation UT as,
\begin{align}
UT:\mathbb{R}^3\rightarrow\mathbb{R}^3
\end{align}
And the transformation in matrix form where the dimension of transformation matrix $\vec{C}$ is 3x3 can be written as,
\begin{align}
UT(\vec{x})=\vec{C}\vec{x}\\
U(T(\vec{x}))=\vec{B}(\vec{A}\vec{x})\\
\implies \vec{C}=\vec{B}\vec{A} \label{MatC}\\
dim(\vec{C})=3\times3 \label{dimC}
\end{align}
Calculating the maximum rank of transformation matrix $\vec{C}$ can have, we know
\begin{align}
Rank(\vec{C})\le min(Rank(\vec{B}),Rank(\vec{A}))
\end{align}
From the equations, \eqref{eqA} and \eqref{eqB}, we get
\begin{align}
Rank(C)\le min(2,2)\\
\implies Max(Rank(C))=2 \label{eqC}
\end{align}
From equations \eqref{dimC} and \eqref{eqC},
\begin{align}
Rank(\vec{C})< dim(\vec{C})
\end{align}
Therefore transformation UT is not invertible.
\section{Theorem}
Generalizing the proof, for $n>m$ and considering vectors $\vec{v},\vec{x}\in\mathbb{R}^n$ and $\vec{w}\in\mathbb{R}^m$. From the Table \ref{table:1}, 
\begin{align}
Rank(\vec{C})=m\\
dim(\vec{C})=n\\
\implies Rank(\vec{C})<dim(\vec{C})\label{prove}
\end{align}
From equation \eqref{prove}we can say that the transformation UT is not invertible. 
\begin{table*}[h!]
\begin{center}
\begin{tabular}{|l|l|l|l|}
\hline
\textbf{Transformation}&\textbf{Matrix Representation}&\textbf{Dimension}&\textbf{Max Rank  of transformation matrix}\\[0.5ex]
\hline
$T:\mathbb{R}^n\rightarrow\mathbb{R}^m$ & $T(\vec{v})=\vec{A}\vec{v}$ & $\vec{A}:m\times n$ & $Rank(\vec{A})=m$\\[0.5ex]
\hline
$U:\mathbb{R}^m\rightarrow\mathbb{R}^n$ & $U(\vec{w})=\vec{B}\vec{w}$ & $\vec{B}:n\times m$ & $Rank(\vec{B})=m$\\[0.5ex]
\hline
$UT:\mathbb{R}^n\rightarrow\mathbb{R}^n$ & $UT(\vec{x})=\vec{C}\vec{x}$ & $\vec{C}:n\times n$ & $Rank(\vec{C})\le min(Rank(\vec{B}),Rank(\vec{A}))$\\[0.5ex]
&&&$Rank(\vec{C})=m$\\[0.5ex]
\hline
\end{tabular}
\caption{Generalization of the proof }
\label{table:1}
\end{center}
\vspace{-0.5cm}
\end{table*}
\section{Example}
Let the vectors $\vec{v}=\myvec{3\\2\\1} \in \mathbb{R}^3$ and $\vec{w}=\myvec{4\\2} \in \mathbb{R}^2$.
\begin{enumerate}
\item Calculating transformation matrix $\vec{A}$, from equation \eqref{Tmat}, 
\begin{align}
T(\vec{v})=\vec{A}\vec{v}\\
\myvec{4\\2}=\myvec{1&0&1\\1&0&-1}\myvec{3\\2\\1}\\
\myvec{1&0&1\\1&0&-1} \sim \myvec{1&0&1\\0&0&2}=rref(\vec{A})\\
\implies Rank(\vec{A})=2 \label{RankA}
\end{align}
\item Calculating transformation matrix $\vec{B}$, from equation \eqref{Umat}, 
\begin{align}
U(\vec{w})=\vec{B}\vec{w}\\
\myvec{3\\2\\1}=\myvec{\frac{3}{4}&2\\1&-1\\0&\frac{1}{2}}\myvec{4\\2}\\
\myvec{\frac{3}{4}&2\\1&-1\\0&\frac{1}{2}} \sim \myvec{\frac{3}{4}&0\\0&-1\\0&0}=rref(\vec{B})\\
\implies Rank(\vec{B})=2 \label{RankB}
\end{align}
\item Now for the transformation UT, calculating the transformation matrix $\vec{C}$, from the equation \eqref{MatC},
\begin{align}
\vec{C}=\vec{B}\vec{A}
\end{align}
\begin{align}
\vec{C}=\myvec{\frac{3}{4}&2\\1&-1\\0&\frac{1}{2}}\myvec{1&0&1\\1&0&-1}=\myvec{\frac{3}{4}&0&\frac{3}{4}\\0&0&2\\\frac{1}{2}&0&\frac{-1}{2}}\\
\myvec{\frac{3}{4}&0&\frac{3}{4}\\0&0&2\\\frac{1}{2}&0&\frac{-1}{2}} \sim \myvec{\frac{3}{4}&0&\frac{3}{4}\\0&0&2\\0&0&0}=rref(\vec{C})
\end{align}
\begin{align}
\implies Rank(\vec{C})=2 \label{RankC}\\
dim(\vec{C})=3
\end{align}
As $Rank(\vec{C})<dim(\vec{C})$, transformation UT (matrix $\vec{C}$) is not invertible.
\end{enumerate}
\end{document}
\end{document}