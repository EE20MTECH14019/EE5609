\documentclass[journal,12pt,twocolumn]{IEEEtran}

\usepackage{setspace}
\usepackage{gensymb}

\singlespacing
\usepackage[cmex10]{amsmath}

\usepackage{amsthm}
\usepackage{mathrsfs}
\usepackage{txfonts}
\usepackage{stfloats}
\usepackage{bm}
\usepackage{cite}
\usepackage{cases}
\usepackage{subfig}
\usepackage{float}
\usepackage{longtable}
\usepackage{multirow}
\usepackage{caption}
%\usepackage[font=bf,labelfont=bf]{caption}

\usepackage{enumitem}
\usepackage{mathtools}
\usepackage{steinmetz}
\usepackage{tikz}
\usepackage{circuitikz}
\usepackage{verbatim}
\usepackage{tfrupee}
\usepackage[breaklinks=true]{hyperref}

\usepackage{tkz-euclide}

\usetikzlibrary{calc,math}
\usepackage{listings}
    \usepackage{color}                                            %%
    \usepackage{array}                                            %%
    \usepackage{longtable}                                        %%
    \usepackage{calc}                                             %%
    \usepackage{multirow}                                         %%
    \usepackage{hhline}                                           %%
    \usepackage{ifthen}                                           %%
    \usepackage{lscape}     
\usepackage{multicol}
\usepackage{chngcntr}


\DeclareMathOperator*{\Res}{Res}

\renewcommand\thesection{\arabic{section}}
\renewcommand\thesubsection{\thesection.\arabic{subsection}}
\renewcommand\thesubsubsection{\thesubsection.\arabic{subsubsection}}

\renewcommand\thesectiondis{\arabic{section}}
\renewcommand\thesubsectiondis{\thesectiondis.\arabic{subsection}}
\renewcommand\thesubsubsectiondis{\thesubsectiondis.\arabic{subsubsection}}
\numberwithin{table}{section}

\hyphenation{op-tical net-works semi-conduc-tor}
\def\inputGnumericTable{}                                 %%

\lstset{
%language=C,
frame=single, 
breaklines=true,
columns=fullflexible
}
\makeatletter
\renewcommand*\env@matrix[1][*\c@MaxMatrixCols c]{%
  \hskip -\arraycolsep
  \let\@ifnextchar\new@ifnextchar
  \array{#1}}
\makeatother
\begin{document}


\newtheorem{theorem}{Theorem}[section]
\newtheorem{problem}{Problem}
\newtheorem{proposition}{Proposition}[section]
\newtheorem{lemma}{Lemma}[section]
\newtheorem{corollary}[theorem]{Corollary}
\newtheorem{example}{Example}[section]
\newtheorem{definition}[problem]{Definition}

\newcommand{\BEQA}{\begin{eqnarray}}
\newcommand{\EEQA}{\end{eqnarray}}
\newcommand{\define}{\stackrel{\triangle}{=}}
\bibliographystyle{IEEEtran}
\providecommand{\mbf}{\mathbf}
\providecommand{\pr}[1]{\ensuremath{\Pr\left(#1\right)}}
\providecommand{\qfunc}[1]{\ensuremath{Q\left(#1\right)}}
\providecommand{\sbrak}[1]{\ensuremath{{}\left[#1\right]}}
\providecommand{\lsbrak}[1]{\ensuremath{{}\left[#1\right.}}
\providecommand{\rsbrak}[1]{\ensuremath{{}\left.#1\right]}}
\providecommand{\brak}[1]{\ensuremath{\left(#1\right)}}
\providecommand{\lbrak}[1]{\ensuremath{\left(#1\right.}}
\providecommand{\rbrak}[1]{\ensuremath{\left.#1\right)}}
\providecommand{\cbrak}[1]{\ensuremath{\left\{#1\right\}}}
\providecommand{\lcbrak}[1]{\ensuremath{\left\{#1\right.}}
\providecommand{\rcbrak}[1]{\ensuremath{\left.#1\right\}}}
\theoremstyle{remark}
\newtheorem{rem}{Remark}
\newcommand{\sgn}{\mathop{\mathrm{sgn}}}
\providecommand{\abs}[1]{\left\vert#1\right\vert}
\providecommand{\res}[1]{\Res\displaylimits_{#1}} 
\providecommand{\norm}[1]{\left\lVert#1\right\rVert}
%\providecommand{\norm}[1]{\lVert#1\rVert}
\providecommand{\mtx}[1]{\mathbf{#1}}
\providecommand{\mean}[1]{E\left[ #1 \right]}
\providecommand{\fourier}{\overset{\mathcal{F}}{ \rightleftharpoons}}
%\providecommand{\hilbert}{\overset{\mathcal{H}}{ \rightleftharpoons}}
\providecommand{\system}{\overset{\mathcal{H}}{ \longleftrightarrow}}
	%\newcommand{\solution}[2]{\textbf{Solution:}{#1}}
\newcommand{\solution}{\noindent \textbf{Solution: }}
\newcommand{\cosec}{\,\text{cosec}\,}
\providecommand{\dec}[2]{\ensuremath{\overset{#1}{\underset{#2}{\gtrless}}}}
\newcommand{\myvec}[1]{\ensuremath{\begin{pmatrix}#1\end{pmatrix}}}
\newcommand{\mydet}[1]{\ensuremath{\begin{vmatrix}#1\end{vmatrix}}}
\numberwithin{equation}{subsection}
\makeatletter
\@addtoreset{figure}{problem}
\makeatother
\let\StandardTheFigure\thefigure
\let\vec\mathbf
\renewcommand{\thefigure}{\theproblem}
\def\putbox#1#2#3{\makebox[0in][l]{\makebox[#1][l]{}\raisebox{\baselineskip}[0in][0in]{\raisebox{#2}[0in][0in]{#3}}}}
     \def\rightbox#1{\makebox[0in][r]{#1}}
     \def\centbox#1{\makebox[0in]{#1}}
     \def\topbox#1{\raisebox{-\baselineskip}[0in][0in]{#1}}
     \def\midbox#1{\raisebox{-0.5\baselineskip}[0in][0in]{#1}}
\vspace{3cm}
\title{Assignment 16}
\author{Yenigalla Samyuktha\\EE20MTECH14019}
\maketitle
\newpage
\bigskip
\renewcommand{\thefigure}{\theenumi}
\renewcommand{\thetable}{1}
\setlength{\tabcolsep}{20pt}
\renewcommand{\arraystretch}{1.5}
\begin{abstract}
This document solves a problem on Jordan form of a complex matrix.
\end{abstract}
Download all latex-tikz codes from 
%
\begin{lstlisting}
https://github.com/EE20MTECH14019/EE5609/tree/master/Assignment_16
\end{lstlisting}
%
\section{Problem}
How many possible Jordan forms are there for a $6\times6$ complex matrix with characteristic polynomial $\brak{x+2}^4\brak{x-1}^2$?
\section{Solution}
\begin{table}[ht!]
\begin{center}
\begin{tabular}{|p{2cm}|p{5.5cm}|}
\hline
\textbf{Parameter} & \textbf{Description}
\\ [0.5ex] 
\hline
$A_M$ & Algebraic multiplicity of characteristic value $\lambda$ in the characteristic polynomial, also equal to the size of Jordan block for that eigen value
\\ [0.5ex] 
\hline
$G_M$ & Geometric multiplicity determines the number of Jordan sub-blocks in a Jordan block for $\lambda$.
\\ [0.5ex] 
\hline
$\vec{J}_{\brak{x-\lambda}^k}$ & Jordan block corresponding to the eigen value $\lambda$ and k is the multiplicity of $\lambda$ in the minimal polynomial determines size of largest Jordan sub-block.
\\ [0.5ex] 
\hline
\end{tabular}
\caption{Parameters}
\label{table:1}
\end{center}
\vspace{-0.5cm}
\end{table}

\renewcommand{\thetable}{2}
\begin{table*}[ht!]
\begin{center}
\begin{tabular}{|p{5cm}|p{11cm}|}
\hline
\textbf{Feature} & \textbf{Explanation}
\\ [0.5ex] 
\hline
Characteristic Polynomial & \parbox{10cm}{\begin{align} \brak{x+2}^4\brak{x-1}^2\label{eq:cp} \end{align}}
\\ [0.5ex] 
\hline
Algebraic Multiplicity, $A_M$ &  \parbox{10cm}{\begin{align} \text{For }\lambda=-2, A_M=4\\ \text{For }\lambda=1, A_M=2 \end{align}}
\\ [0.5ex]
\hline
Minimal Polynomial & \parbox{10cm}{\begin{align} p=\brak{x+2}^a\brak{x-1}^b \; ,a\le 4,b\le 2\label{eq:mp} \end{align}}
\\ [0.5ex] 
\hline
Possibilities of minimal polynomial & From equation \eqref{eq:mp}, there are 8 different minimal polynomials possible.
\\ [0.5ex] 
\hline
Jordan Form & Jordan form built from Jordan blocks listed in table \ref{table:3} will have the form,\\ & \parbox{10cm}{\begin{align} \vec{J}=\myvec{-2&*&0&0&0&0\\0&-2&*&0&0&0\\0&0&-2&*&0&0\\0&0&0&-2&*&0\\0&0&0&0&1&*\\0&0&0&0&0&1} \end{align}}
\\ [0.5ex]
& where $*$ can be either 1 or 0
\\ [0.5ex] 
\hline
Number of possibilities of Jordan canonical forms & From Table \ref{table:3}, there are $5\times2=10$ different Jordan forms possible.
\\ [0.5ex] 
\hline
Jordan Form corresponding to $p=\brak{x+2}^2\brak{x-1}$ & One minimal polynomial can correspond to more than one Jordan forms. For example, minimal polynomial $p=\brak{x+2}^2\brak{x-1}$ can correspond to two different Jordan forms namely,
\\ [0.5ex] 
 & \parbox{10cm}{\begin{align} \vec{J}=\myvec{-2&1&0&0&0&0\\0&-2&0&0&0&0\\0&0&-2&1&0&0\\0&0&0&-2&0&0\\0&0&0&0&1&0\\0&0&0&0&0&1}, \quad \vec{J}=\myvec{-2&1&0&0&0&0\\0&-2&0&0&0&0\\0&0&-2&0&0&0\\0&0&0&-2&0&0\\0&0&0&0&1&0\\0&0&0&0&0&1} \end{align}}
\\ [0.5ex] 
\hline
\end{tabular}
\caption{Parameters}
\label{table:2}
\end{center}
\vspace{-0.5cm}
\end{table*}

\renewcommand{\thetable}{3}
\begin{table*}[ht!]
\begin{center}
\begin{tabular}{|c|c|c|}
\hline
\textbf{Factor} & \textbf{Possible Jordan blocks} & $G_M$ \\[0.5ex]
\hline
\multirow{5}{*}{$\brak{x+2}$} & 
$\vec{J}_{\brak{x+2}}=\myvec{-2&0&0&0\\0&-2&0&0\\0&0&-2&0\\0&0&0&-2}$ & 4\\ [0.5ex]  \cline{2-3}
& $\vec{J}_{\brak{x+2}^2}=\myvec{-2&1&0&0\\0&-2&0&0\\0&0&-2&0\\0&0&0&-2}$ & 3\\  [0.5ex] \cline{2-3}
& $\vec{J}_{\brak{x+2}^2}=\myvec{-2&1&0&0\\0&-2&0&0\\0&0&-2&1\\0&0&0&-2}$ & 2\\  [0.5ex] \cline{2-3}
& $\vec{J}_{\brak{x+2}^3}=\myvec{-2&1&0&0\\0&-2&1&0\\0&0&-2&0\\0&0&0&-2}$ & 2\\ [0.5ex]  \cline{2-3}
& $\vec{J}_{\brak{x+2}^4}=\myvec{-2&1&0&0\\0&-2&1&0\\0&0&-2&1\\0&0&0&-2}$ & 1
\\ [0.5ex] 
\hline
\multirow{2}{*}{$\brak{x-1}$} & 
$\vec{J}_{\brak{x-1}}=\myvec{1&0\\0&1}$ &  2\\ [0.5ex]  \cline{2-3}
& $\vec{J}_{\brak{x-1}^2}=\myvec{1&1\\0&1}$ & 1
\\ [0.5ex] 
\hline
\end{tabular}
\caption{Possible Jordan Blocks}
\label{table:3}
\end{center}
\vspace{-0.5cm}
\end{table*}
\end{document}