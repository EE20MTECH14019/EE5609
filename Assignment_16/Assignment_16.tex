\documentclass[journal,12pt,twocolumn]{IEEEtran}

\usepackage{setspace}
\usepackage{gensymb}

\singlespacing
\usepackage[cmex10]{amsmath}

\usepackage{amsthm}
\usepackage{mathrsfs}
\usepackage{txfonts}
\usepackage{stfloats}
\usepackage{bm}
\usepackage{cite}
\usepackage{cases}
\usepackage{subfig}
\usepackage{float}
\usepackage{longtable}
\usepackage{multirow}
\usepackage{caption}
%\usepackage[font=bf,labelfont=bf]{caption}

\usepackage{enumitem}
\usepackage{mathtools}
\usepackage{steinmetz}
\usepackage{tikz}
\usepackage{circuitikz}
\usepackage{verbatim}
\usepackage{tfrupee}
\usepackage[breaklinks=true]{hyperref}

\usepackage{tkz-euclide}

\usetikzlibrary{calc,math}
\usepackage{listings}
    \usepackage{color}                                            %%
    \usepackage{array}                                            %%
    \usepackage{longtable}                                        %%
    \usepackage{calc}                                             %%
    \usepackage{multirow}                                         %%
    \usepackage{hhline}                                           %%
    \usepackage{ifthen}                                           %%
    \usepackage{lscape}     
\usepackage{multicol}
\usepackage{chngcntr}


\DeclareMathOperator*{\Res}{Res}

\renewcommand\thesection{\arabic{section}}
\renewcommand\thesubsection{\thesection.\arabic{subsection}}
\renewcommand\thesubsubsection{\thesubsection.\arabic{subsubsection}}

\renewcommand\thesectiondis{\arabic{section}}
\renewcommand\thesubsectiondis{\thesectiondis.\arabic{subsection}}
\renewcommand\thesubsubsectiondis{\thesubsectiondis.\arabic{subsubsection}}
\numberwithin{table}{section}

\hyphenation{op-tical net-works semi-conduc-tor}
\def\inputGnumericTable{}                                 %%

\lstset{
%language=C,
frame=single, 
breaklines=true,
columns=fullflexible
}
\makeatletter
\renewcommand*\env@matrix[1][*\c@MaxMatrixCols c]{%
  \hskip -\arraycolsep
  \let\@ifnextchar\new@ifnextchar
  \array{#1}}
\makeatother
\begin{document}


\newtheorem{theorem}{Theorem}[section]
\newtheorem{problem}{Problem}
\newtheorem{proposition}{Proposition}[section]
\newtheorem{lemma}{Lemma}[section]
\newtheorem{corollary}[theorem]{Corollary}
\newtheorem{example}{Example}[section]
\newtheorem{definition}[problem]{Definition}

\newcommand{\BEQA}{\begin{eqnarray}}
\newcommand{\EEQA}{\end{eqnarray}}
\newcommand{\define}{\stackrel{\triangle}{=}}
\bibliographystyle{IEEEtran}
\providecommand{\mbf}{\mathbf}
\providecommand{\pr}[1]{\ensuremath{\Pr\left(#1\right)}}
\providecommand{\qfunc}[1]{\ensuremath{Q\left(#1\right)}}
\providecommand{\sbrak}[1]{\ensuremath{{}\left[#1\right]}}
\providecommand{\lsbrak}[1]{\ensuremath{{}\left[#1\right.}}
\providecommand{\rsbrak}[1]{\ensuremath{{}\left.#1\right]}}
\providecommand{\brak}[1]{\ensuremath{\left(#1\right)}}
\providecommand{\lbrak}[1]{\ensuremath{\left(#1\right.}}
\providecommand{\rbrak}[1]{\ensuremath{\left.#1\right)}}
\providecommand{\cbrak}[1]{\ensuremath{\left\{#1\right\}}}
\providecommand{\lcbrak}[1]{\ensuremath{\left\{#1\right.}}
\providecommand{\rcbrak}[1]{\ensuremath{\left.#1\right\}}}
\theoremstyle{remark}
\newtheorem{rem}{Remark}
\newcommand{\sgn}{\mathop{\mathrm{sgn}}}
\providecommand{\abs}[1]{\left\vert#1\right\vert}
\providecommand{\res}[1]{\Res\displaylimits_{#1}} 
\providecommand{\norm}[1]{\left\lVert#1\right\rVert}
%\providecommand{\norm}[1]{\lVert#1\rVert}
\providecommand{\mtx}[1]{\mathbf{#1}}
\providecommand{\mean}[1]{E\left[ #1 \right]}
\providecommand{\fourier}{\overset{\mathcal{F}}{ \rightleftharpoons}}
%\providecommand{\hilbert}{\overset{\mathcal{H}}{ \rightleftharpoons}}
\providecommand{\system}{\overset{\mathcal{H}}{ \longleftrightarrow}}
	%\newcommand{\solution}[2]{\textbf{Solution:}{#1}}
\newcommand{\solution}{\noindent \textbf{Solution: }}
\newcommand{\cosec}{\,\text{cosec}\,}
\providecommand{\dec}[2]{\ensuremath{\overset{#1}{\underset{#2}{\gtrless}}}}
\newcommand{\myvec}[1]{\ensuremath{\begin{pmatrix}#1\end{pmatrix}}}
\newcommand{\mydet}[1]{\ensuremath{\begin{vmatrix}#1\end{vmatrix}}}
\numberwithin{equation}{subsection}
\makeatletter
\@addtoreset{figure}{problem}
\makeatother
\let\StandardTheFigure\thefigure
\let\vec\mathbf
\renewcommand{\thefigure}{\theproblem}
\def\putbox#1#2#3{\makebox[0in][l]{\makebox[#1][l]{}\raisebox{\baselineskip}[0in][0in]{\raisebox{#2}[0in][0in]{#3}}}}
     \def\rightbox#1{\makebox[0in][r]{#1}}
     \def\centbox#1{\makebox[0in]{#1}}
     \def\topbox#1{\raisebox{-\baselineskip}[0in][0in]{#1}}
     \def\midbox#1{\raisebox{-0.5\baselineskip}[0in][0in]{#1}}
\vspace{3cm}
\title{Assignment 16}
\author{Yenigalla Samyuktha\\EE20MTECH14019}
\maketitle
\newpage
\bigskip
\renewcommand{\thefigure}{\theenumi}
\renewcommand{\thetable}{1}
\setlength{\tabcolsep}{20pt}
\renewcommand{\arraystretch}{1.5}
\begin{abstract}
This document solves a problem on Jordan form of a complex matrix.
\end{abstract}
Download all latex-tikz codes from 
%
\begin{lstlisting}
https://github.com/EE20MTECH14019/EE5609/tree/master/Assignment_16
\end{lstlisting}
%
\section{Problem}
How many possible Jordan forms are there for a $6\times6$ complex matrix with characteristic polynomial $\brak{x+2}^4\brak{x-1}^2$?
\section{Explanation}
From the characteristic polynomial,
\begin{align}
\brak{x+2}^4\brak{x-1}^2\label{eq:p}
\end{align}
We get the eigen values of the $6\times6$ complex matrix as,
\begin{align}
\lambda_i=\cbrak{-2,-2,-2,-2,1,1}\label{eq:eig}
\end{align}
The minimal polynomial for a matrix with characteristic polynomial must have both $\brak{x+2}$ and $\brak{x-1}$ as factors and it must divide \eqref{eq:p}. Hence the minimal polynomial will be of the form 
\begin{align}
p=\brak{x+2}^a\brak{x-1}^b \; ,a\le 4,b\le2
\end{align}
So, there are 8 different possibilities for a minimal polynomial. Let us note that one minimal polynomial may correspond to more than one Jordan forms. The possible Jordan blocks associated with the eigen values -2 and 1 are given in the Table \ref{table:1}. From Table \ref{table:1} , we can say that there are $5\times2=10$ possible Jordan forms.
For example, the Jordan forms corresponding to minimal polynomial $p=\brak{x+2}^2\brak{x-1}$ built from the Jordan blocks can be,
\begin{align}
\vec{J}=\myvec{-2&1&0&0&0&0\\0&-2&0&0&0&0\\0&0&-2&1&0&0\\0&0&0&-2&0&0\\0&0&0&0&1&0\\0&0&0&0&0&1}
\end{align}
and
\begin{align}
\vec{J}=\myvec{-2&1&0&0&0&0\\0&-2&0&0&0&0\\0&0&-2&0&0&0\\0&0&0&-2&0&0\\0&0&0&0&1&0\\0&0&0&0&0&1}
\end{align}
\renewcommand{\thetable}{1}
\begin{table*}[ht!]
\begin{center}
\begin{tabular}{|l|l|}
\hline
\textbf{Factor} & \textbf{Possible Jordan blocks} \\[0.5ex]
\hline
 $\brak{x+2}$ & \parbox{10cm}{\begin{align}
\myvec{-2&0&0&0\\0&-2&0&0\\0&0&-2&0\\0&0&0&-2}\\
\myvec{-2&1&0&0\\0&-2&0&0\\0&0&-2&0\\0&0&0&-2}\\
\myvec{-2&1&0&0\\0&-2&0&0\\0&0&-2&1\\0&0&0&-2},\myvec{-2&1&0&0\\0&-2&1&0\\0&0&-2&0\\0&0&0&-2}\\
\myvec{-2&1&0&0\\0&-2&1&0\\0&0&-2&1\\0&0&0&-2}
\end{align}}
\\ [0.5ex] 
\hline
$\brak{x-1}$ & \parbox{10cm}{\begin{align}
\myvec{1&0\\0&1}\\\myvec{1&1\\0&1}
\end{align}}
\\ [0.5ex] 
\hline
\end{tabular}
\caption{Possible Jordan Blocks}
\label{table:1}
\end{center}
\vspace{-0.5cm}
\end{table*}
\section{Answer}
Therefore 10 different Jordan forms are possible for a $6\times6$ complex matrix with characteristic polynomial $\brak{x+2}^4\brak{x-1}^2$.
\end{document}