\documentclass[journal,12pt,twocolumn]{IEEEtran}

\usepackage{setspace}
\usepackage{gensymb}

\singlespacing


\usepackage[cmex10]{amsmath}

\usepackage{amsthm}

\usepackage{mathrsfs}
\usepackage{txfonts}
\usepackage{stfloats}
\usepackage{bm}
\usepackage{cite}
\usepackage{cases}
\usepackage{subfig}

\usepackage{longtable}
\usepackage{multirow}

\usepackage{enumitem}
\usepackage{mathtools}
\usepackage{steinmetz}
\usepackage{tikz}
\usepackage{circuitikz}
\usepackage{verbatim}
\usepackage{tfrupee}
\usepackage[breaklinks=true]{hyperref}

\usepackage{tkz-euclide}

\usetikzlibrary{calc,math}
\usepackage{listings}
    \usepackage{color}                                            %%
    \usepackage{array}                                            %%
    \usepackage{longtable}                                        %%
    \usepackage{calc}                                             %%
    \usepackage{multirow}                                         %%
    \usepackage{hhline}                                           %%
    \usepackage{ifthen}                                           %%
    \usepackage{lscape}     
\usepackage{multicol}
\usepackage{chngcntr}

\DeclareMathOperator*{\Res}{Res}

\renewcommand\thesection{\arabic{section}}
\renewcommand\thesubsection{\thesection.\arabic{subsection}}
\renewcommand\thesubsubsection{\thesubsection.\arabic{subsubsection}}

\renewcommand\thesectiondis{\arabic{section}}
\renewcommand\thesubsectiondis{\thesectiondis.\arabic{subsection}}
\renewcommand\thesubsubsectiondis{\thesubsectiondis.\arabic{subsubsection}}


\hyphenation{op-tical net-works semi-conduc-tor}
\def\inputGnumericTable{}                                 %%

\lstset{
%language=C,
frame=single, 
breaklines=true,
columns=fullflexible
}
\begin{document}


\newtheorem{theorem}{Theorem}[section]
\newtheorem{problem}{Problem}
\newtheorem{proposition}{Proposition}[section]
\newtheorem{lemma}{Lemma}[section]
\newtheorem{corollary}[theorem]{Corollary}
\newtheorem{example}{Example}[section]
\newtheorem{definition}[problem]{Definition}

\newcommand{\BEQA}{\begin{eqnarray}}
\newcommand{\EEQA}{\end{eqnarray}}
\newcommand{\define}{\stackrel{\triangle}{=}}
\bibliographystyle{IEEEtran}
\providecommand{\mbf}{\mathbf}
\providecommand{\pr}[1]{\ensuremath{\Pr\left(#1\right)}}
\providecommand{\qfunc}[1]{\ensuremath{Q\left(#1\right)}}
\providecommand{\sbrak}[1]{\ensuremath{{}\left[#1\right]}}
\providecommand{\lsbrak}[1]{\ensuremath{{}\left[#1\right.}}
\providecommand{\rsbrak}[1]{\ensuremath{{}\left.#1\right]}}
\providecommand{\brak}[1]{\ensuremath{\left(#1\right)}}
\providecommand{\lbrak}[1]{\ensuremath{\left(#1\right.}}
\providecommand{\rbrak}[1]{\ensuremath{\left.#1\right)}}
\providecommand{\cbrak}[1]{\ensuremath{\left\{#1\right\}}}
\providecommand{\lcbrak}[1]{\ensuremath{\left\{#1\right.}}
\providecommand{\rcbrak}[1]{\ensuremath{\left.#1\right\}}}
\theoremstyle{remark}
\newtheorem{rem}{Remark}
\newcommand{\sgn}{\mathop{\mathrm{sgn}}}
\providecommand{\abs}[1]{\left\vert#1\right\vert}
\providecommand{\res}[1]{\Res\displaylimits_{#1}} 
\providecommand{\norm}[1]{\left\lVert#1\right\rVert}
%\providecommand{\norm}[1]{\lVert#1\rVert}
\providecommand{\mtx}[1]{\mathbf{#1}}
\providecommand{\mean}[1]{E\left[ #1 \right]}
\providecommand{\fourier}{\overset{\mathcal{F}}{ \rightleftharpoons}}
%\providecommand{\hilbert}{\overset{\mathcal{H}}{ \rightleftharpoons}}
\providecommand{\system}{\overset{\mathcal{H}}{ \longleftrightarrow}}
	%\newcommand{\solution}[2]{\textbf{Solution:}{#1}}
\newcommand{\solution}{\noindent \textbf{Solution: }}
\newcommand{\cosec}{\,\text{cosec}\,}
\providecommand{\dec}[2]{\ensuremath{\overset{#1}{\underset{#2}{\gtrless}}}}
\newcommand{\myvec}[1]{\ensuremath{\begin{pmatrix}#1\end{pmatrix}}}
\newcommand{\mydet}[1]{\ensuremath{\begin{vmatrix}#1\end{vmatrix}}}
\numberwithin{equation}{subsection}
\makeatletter
\@addtoreset{figure}{problem}
\makeatother
\let\StandardTheFigure\thefigure
\let\vec\mathbf
\renewcommand{\thefigure}{\theproblem}
\def\putbox#1#2#3{\makebox[0in][l]{\makebox[#1][l]{}\raisebox{\baselineskip}[0in][0in]{\raisebox{#2}[0in][0in]{#3}}}}
     \def\rightbox#1{\makebox[0in][r]{#1}}
     \def\centbox#1{\makebox[0in]{#1}}
     \def\topbox#1{\raisebox{-\baselineskip}[0in][0in]{#1}}
     \def\midbox#1{\raisebox{-0.5\baselineskip}[0in][0in]{#1}}
\vspace{3cm}
\title{Assignment 6}
\author{Yenigalla Samyuktha}
\maketitle
\newpage
\bigskip
\renewcommand{\thefigure}{\theenumi}
\renewcommand{\thetable}{\theenumi}
\begin{abstract}
This document explains the process of finding the distance between a given point and a plane using Singular Value Decomposition.
\end{abstract}
Download all python codes from 
\begin{lstlisting}
https://github.com/EE20MTECH14019/EE5609/tree/master/Assignment_6/codes
\end{lstlisting}
%
and latex-tikz codes from 
%
\begin{lstlisting}
https://github.com/EE20MTECH14019/EE5609/tree/master/Assignment_6
\end{lstlisting}
%
\section{Problem}
Find the foot of perpendicular from the point $\vec{A}=\myvec{-3\\2\\1}$ on the plane $\myvec{3&2&-6}\vec{x} = 2$.
\section{Explanation}
Consider orthogonal vectors $\vec{m_1}$ and $\vec{m_2}$ to the given normal vector $\vec{n}$. Let, $\vec{m}$ = $\myvec{a\\b\\c}$, then
\begin{align}
\vec{m^T}\vec{n} &= 0\\
\implies\myvec{a&b&c}\myvec{3\\2\\-6} &= 0\\
\implies3a+2b-6c &= 0
\end{align}
Let a=1 and b=0 we get,
\begin{align}
\vec{m_1} &= \myvec{1\\0\\\frac{1}{2}} \label{eq:eq1}
\end{align}
Let a=0 and b=1 we get,
\begin{align}
\vec{m_2} &= \myvec{0\\1\\\frac{1}{3}} \label{eq:eq2}
\end{align}
Solving the equation,
\begin{align}
\vec{M}\vec{x} &= \vec{b}\label{eq:eq3}
\end{align}
Substituting \eqref{eq:eq1} and \eqref{eq:eq2} in \eqref{eq:eq3},
\begin{align}
    \myvec{1&0\\0&1\\\frac{1}{2}&\frac{1}{3}}\vec{x} &= \myvec{-3\\2\\1}\label{eq:eq4}
\end{align}
Solving \eqref{eq:eq4} using Singular Value Decomposition on $\vec{M}$ as follows,
\begin{align}
\vec{M}=\vec{U}\vec{\Sigma}\vec{V}^T\label{eq:eq5}
\end{align}
Where the columns of $\vec{V}$ are the eigen vectors of $\vec{M}^T\vec{M}$ ,the columns of $\vec{U}$ are the eigen vectors of $\vec{M}\vec{M}^T$ and $\vec{S}$ is diagonal matrix of singular value of eigenvalues of $\vec{M}^T\vec{M}$. We have,
\begin{align}
\vec{M}^T\vec{M}=\myvec{\frac{5}{4}&\frac{1}{6}\\\frac{1}{6}&\frac{10}{9}}\label{eq:eq6}\\
\vec{M}\vec{M}^T=\myvec{1&0&\frac{1}{2}\\0&1&\frac{1}{3}\\\frac{1}{2}&\frac{1}{3}&\frac{13}{36}}
\end{align}
Substituting \eqref{eq:eq5} in \eqref{eq:eq3},
\begin{align}
\vec{U}\vec{\Sigma}\vec{V}^T\vec{x} & = \vec{b}\\
\implies\vec{x} &= \vec{V}\vec{\Sigma^{-1}}\vec{U^T}\vec{b}\label{eq:eq7}
\end{align}
Where $\vec{\Sigma^{-1}}$ is Moore-Penrose Pseudo-Inverse of $\vec{\Sigma}$ and is obtained by inversing only non-zero elements in $\vec{\Sigma}$\\
Calculating eigen values of $\vec{M}\vec{M}^T$,
\begin{align}
\mydet{\vec{M}\vec{M}^T - \lambda\vec{I}} &= 0\\
\implies\mydet{1-\lambda&0&\frac{1}{2}\\0&1-\lambda&\frac{1}{3}\\\frac{1}{2}&\frac{1}{3}&\frac{13}{36}-\lambda} &=0\\
\implies \lambda^3-\frac{85}{36}\lambda^2+\frac{49}{36}\lambda&=0 \label{eq:eq8}
\end{align}
From the characteristic equation \eqref{eq:eq8}, the eigen values of $\vec{M}\vec{M}^T$ are,
\begin{align}
\lambda_1 = \frac{49}{36} \quad
\lambda_2 = 1 \quad
\lambda_3 = 0
\end{align}
The eigen vectors of $\vec{M}\vec{M}^T$ are,
\begin{align}
\vec{u_1}=\myvec{\frac{18}{13}\\\frac{12}{13}\\1}\quad
\vec{u_2}=\myvec{\frac{-2}{3}\\1\\0}\quad
\vec{u_3}=\myvec{\frac{-1}{2}\\\frac{-1}{3}\\1}\label{eq:eq9}
\end{align}
Normalizing the eigen vectors in equation \eqref{eq:eq9}
\begin{align}
\vec{u_1}=\myvec{\frac{18}{7\sqrt{13}}\\\frac{12}{7\sqrt{13}}\\\frac{\sqrt{13}}{7}}\quad
\vec{u_2}=\myvec{\frac{-2}{\sqrt{13}}\\\frac{3}{\sqrt{13}}\\0}\quad
\vec{u_3}=\myvec{\frac{-7}{12}\\\frac{-7}{18}\\\frac{7}{6}}
\end{align}
Hence we obtain $\vec{U}$ as follows,
\begin{align}
\vec{U}=\myvec{\frac{18}{7\sqrt{13}}&\frac{-2}{\sqrt{13}}&\frac{-7}{12}\\\frac{12}{7\sqrt{13}}&\frac{3}{\sqrt{13}}&\frac{-7}{18}\\\frac{\sqrt{13}}{7}&0&\frac{7}{6}}\label{eq:eq10}
\end{align}
By computing the singular values from eigen values $\lambda_1, \lambda_2, \lambda_3$ we get $\vec{\Sigma}$ as,
\begin{align}
\vec{\Sigma}=\myvec{\frac{49}{36}&0\\0&1\\0&0}
\end{align}
Calculating eigen values of $\vec{M}^T\vec{M}$,
\begin{align}
\mydet{\vec{M}^T\vec{M} - \lambda\vec{I}} &= 0\\
\implies\mydet{\frac{5}{4}-\lambda&\frac{1}{6}\\\frac{1}{6}&\frac{10}{9}-\lambda} &=0\\
\implies\lambda^2-\frac{85}{36}\lambda+\frac{49}{36} &=0
\end{align}
From the characteristic equation, the eigen values of $\vec{M}^T\vec{M}$ are,
\begin{align}
\lambda_1 = \frac{49}{36}\quad
\lambda_2 = 1
\end{align}
Hence the eigen vectors of $\vec{M}^T\vec{M}$ are,
\begin{align}
\vec{v}_1=\myvec{\frac{3}{2}\\1} \quad
\vec{v}_2=\myvec{\frac{-2}{3}\\1}
\end{align}
Normalizing the eigen vectors,
\begin{align}
\vec{v}_1=\myvec{\frac{3}{\sqrt{13}}\\\frac{2}{\sqrt{13}}} \quad
\vec{v}_2=\myvec{\frac{-2}{\sqrt{13}}\\\frac{3}{\sqrt{13}}}
\end{align}
Hence we obtain $\vec{V}$ as,
\begin{align}
\vec{V}=\myvec{\frac{3}{\sqrt{13}}&\frac{-2}{\sqrt{13}}\\\frac{2}{\sqrt{13}}&\frac{3}{\sqrt{13}}}
\end{align}
From \eqref{eq:eq3}, the Singular Value Decomposition of $\vec{M}$ is as follows,
\begin{align}
\vec{M} = \myvec{\frac{18}{7\sqrt{13}}&\frac{-2}{\sqrt{13}}&\frac{-7}{12}\\\frac{12}{7\sqrt{13}}&\frac{3}{\sqrt{13}}&\frac{-7}{18}\\\frac{\sqrt{13}}{7}&0&\frac{7}{6}}\myvec{\frac{49}{36}&0\\0&1\\0&0}\myvec{\frac{3}{\sqrt{13}}&\frac{-2}{\sqrt{13}}\\\frac{2}{\sqrt{13}}&\frac{3}{\sqrt{13}}}^T
\end{align}
And, the Moore-Penrose Pseudo inverse of $\vec{\Sigma}$ is given by,
\begin{align}
\vec{\Sigma^{-1}} = \myvec{\frac{6}{7}&0&0\\0&1&0}
\end{align}
From \eqref{eq:eq7} we get,
\begin{align}
\vec{U}^T\vec{b}&=\myvec{\frac{-17}{7\sqrt{13}}\\\frac{12}{\sqrt{13}}\\\frac{77}{36}}\\
\vec{\Sigma^{-1}}\vec{U}^T\vec{b}&=\myvec{\frac{-102}{49\sqrt{13}}\\\frac{12}{\sqrt{13}}}\\
\vec{x} &= \vec{V}\vec{\Sigma^{-1}}\vec{U}^T\vec{b} &= \myvec{\frac{-114}{49}\\\frac{120}{49}}\label{eq:eq11}
\end{align}
Now we verify the solution \eqref{eq:eq11} using,
\begin{align}
\vec{M}\vec{x}=\vec{b}
\implies\vec{M}^T\vec{M}\vec{x} = \vec{M}^T\vec{b}\label{eqVerify}
\end{align}
On evaluating the R.H.S in \eqref{eqVerify} we get,
\begin{align}
\vec{M}^T\vec{M}\vec{x} &= \myvec{\frac{-5}{2}\\\frac{7}{3}}\\
\implies\myvec{\frac{5}{4}&\frac{1}{6}\\\frac{1}{6}&\frac{10}{9}}\vec{x} &= \myvec{\frac{-5}{2}\\\frac{7}{3}}\label{eq:eq12}
\end{align}
On solving the augmented matrix of \eqref{eq:eq12} we get,
\begin{align}
\myvec{\frac{5}{4}&\frac{1}{6}&\frac{-5}{2}\\\frac{1}{6}&\frac{10}{9}&\frac{7}{3}} &\xleftrightarrow{R_1=\frac{4R_1}{5}}\myvec{1&\frac{2}{15}&-2\\\frac{1}{6}&\frac{10}{9}&\frac{7}{3}}\\
&\xleftrightarrow{R_2=R_2-\frac{R_1}{6}}\myvec{1&\frac{2}{15}&-2\\0&\frac{49}{45}&\frac{8}{3}}\\
&\xleftrightarrow{R_2=\frac{45}{49}R_2}\myvec{1&\frac{2}{15}&-2\\0&1&\frac{120}{49}}\\
&\xleftrightarrow{R_1=R_1-\frac{2R_2}{15}}\myvec{1&0&\frac{-114}{49}\\0&1&\frac{120}{49}}\label{eq:eq13}
\end{align}
From equation \eqref{eq:eq13}, solution is given by,
\begin{align}
\vec{x}=\myvec{\frac{-114}{49}\\\frac{120}{49}}\label{eq:eq14}
\end{align}
From the equations \eqref{eq:eq11} and \eqref{eq:eq14}, the solution $\vec{x}$ is verified.
\\\end{document}